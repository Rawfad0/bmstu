\chapter{Сравнение методов}

Для сравнения методов были сформулированы следующие критерии:
\begin{enumerate}
	\item взаимодействие с ЛПР; % отсутствует, до, после, в процессе
	\item информация о предпочтениях ЛПР; % - + - +
	\item количество взаимодействий с ЛПР; % 0, 1, 1, >0
	\item количество решений, предоставляемых ЛПР. % 1, 1, много, много
\end{enumerate}

В таблице~\ref{t:table} представлено сравнение перечесленных классов методов по сформулированным критериям.


\renewcommand{\arraystretch}{1.5}
\begin{table}[!ht]
	
	\begin{center}
		\small
		\begin{threeparttable}
			\caption{Сравнение классов методов}
			\label{t:table}
			\begin{tabular}{|p{8cm} | m{2cm} | m{1.5cm} | m{1.5cm} | m{1.5cm} | }
				\hline
				\bfseries Классы методов
				& \multicolumn{4}{c|}{\bfseries Номер критерия} \\ 
				\cline{2-5} 
				& \bfseries 1
				& \bfseries 2
				& \bfseries 3
				& \bfseries 4 \\  
				\hline
				Методы, не учитывающие предпочтения & отсутсвует & - & 0 & 1  \\
				\hline
				Априорные & до & + & 1 & 1  \\
				\hline
				Апостериорные & после & - & 1 & много \\
				\hline
				Интерактивные & в процессе & + & >0 & много \\
				\hline
			\end{tabular}	
		\end{threeparttable}
	\end{center}
\end{table}


\clearpage
