\chapter{Анализ предметной области}

\section{Основные определения}

Многокритериальная оптимизация (МКО) --- это одновременная
оптимизация минимум двух (и более) конфликтующих между собой целевых
функций в заданной области определения~\cite{lit3}.

При этом осуществление данного процесса невозможно без лица, которое
принимает конечное решение, им может являться человек или целый коллектив.
Данного человека называют лицом, принимающим решение (ЛПР). Ответственность за выбор того или иного решения и его последствия несет ЛПР~\cite{lit8}.


\section{Формализация задачи}
Задачу многокритериальной оптимизации рассматривают в виде выражения~\ref{eq:1}.
\begin{equation}
	\label{eq:1}
	\min_{X \in D_X}  \Phi(X) =  \Phi(X^*),
\end{equation}
где $X \in \mathbb{R}^n$ – вектор варьируемых параметров; $D_X$ – ограниченное и замкнутое множество допустимых значений этого вектора; $\Phi(X)=(\phi_1(X),...,\phi_m(X))$ – векторный критерий оптимальности. ЛПР стремится найти такой вектор $X^*$ – искомое решение МКО-задачи, который минимизирует на множестве $D_X$ каждый из частных критериев оптимальности. 

Множество, в которое векторный критерий оптимальности $\Phi(X)$ отображает множество $D_X$ , обозначается $D_\Phi$ и называется критериальным множеством (множеством достижимости)~\cite{lit2}. 

Определение отношения доминирования $\triangleright$. $\Phi(X^j)$ обозначается как $Z^j$, $Z^j = (z_1^j,..., z_m^j), X^j \in D_X$, и пишется $Z^1 \triangleright Z^2 $, если $z_i^1 \neq z_i^2$ и среди равенств и неравенств $z_i^1 \le z_i^2, i \in [1:m ]$ имеется хотя бы одно строгое. Вектор $Z^1$ из критериального множества доминирует по Парето вектор $Z^2$ из того же множества, если $Z^1 \triangleright Z^2 $.

Не формально, множество Парето $D_\Phi^*$ поставленной задачи многокритериальной оптимизации определяют как совокупность векторов 
$Z\in D_\Phi$ , среди которых нет доминируемых. Формально, множество Парето определяют выражением~\ref{eq:2}~\cite{lit2}.
\begin{equation}
	\label{eq:2}
	D_\Phi^* = \{\forall Z^* \in D_\Phi | \{Z' \in D_\Phi | Z' \triangleright Z^* \}  = \emptyset \}.
\end{equation} 
Если $\Phi(X^*)\in D_\Phi^*$, то говорится, что $X^*$ – эффективный по Парето вектор. Множество векторов, принадлежащих множеству Парето, обозначается $D_X^*$~\cite{lit2}. 

Интерактивные методы решения задачи многокритериальной оптимизации основаны на гипотезе существования единственной, неизвестной ЛПР скалярной функции его предпочтений $\Psi(X) \in R^1, X \in D_X$. При этом полагают, что большему значению функции $\Psi(X)$ соответствует более предпочтительное с точки зрения ЛПР решение $X$~\cite{lit2}.

%\section*{Вывод}

%В данной части были даны основные определения терминов, использующихся в задаче, и формализована задача многокритериальной оптимизации.

\clearpage
