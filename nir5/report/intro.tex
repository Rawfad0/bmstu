\ssr{ВВЕДЕНИЕ}

Оптимизация представляет собой широко используемый метод исследования, который применяется в различных областях, таких как математика, инженерия, автомобилестроение и другие. Цель оптимизации заключается в максимизации или минимизации функции с учетом ряда ограничений. Однако во многих ситуациях лица, принимающие решения, сталкиваются с необходимостью оптимизации нескольких целевых функций одновременно, что приводит к многокритериальной оптимизации. Если несколько целей не совпадают, эта задача значительно усложняется. 

Теория многокритериальной оптимизации является основой для разработки методов поддержки принятия решений, когда выбор решения осуществляется на основе нескольких критериев~\cite{lit1}. 

Целью данной научно-исследовательской работы является сравнение методов решения задачи многокритериальной оптимизации.

В рамках работы были поставлены следуюдие задачи: 
\begin{enumerate}
	\item провести анализ предметной области;
	\item формализовать задачу многокритериальной оптимизациив;
	\item перечислить существующие методы решения задачи;
	\item сформулированы критерии сравнения методов решения;
	\item провести сравнительный анализ методов решения на основе сформулированных критериев.
\end{enumerate}
\clearpage
