\chapter{Формализация задачи}


Формализация задачи оптимизации в общем виде~\ref{eq:11}.
\begin{equation}
	\label{eq:11}
	\min_{X \in \mathbb{R}^n}  f(X) =  \Phi(X^*), \text{при ограничениях} g(X)  \le 0
\end{equation}
где
$X = (x_1, ..., x_n)$ — вектор переменных (параметров),
$f = (f_1, ..., f_k)$ — векторнозначная целевая функция,
$g = (g_1, ..., g_m)$ — векторнозначная функция ограничений.

Точка $X$, удовлетворяющая всем ограничениям, называется допустимой. Множество всех допустимых точек обозначается $\Omega = \{ X | g(X) \le 0 \}$.

Если $f$ состоит из одной компоненты (скалярная функция), оптимизация называется
однокритериальной, иначе — многокритериальной.

Используя введенные выше обозначения, можно переписать~\ref{eq:11} для многокритериальной оптимизации~\ref{eq:12}.
\begin{equation}
	\label{eq:12}
	(f_1(x), ... , f_k(x)) \rightarrow min, x \in \Omega \text{-- многокритериальная оптимизация}
\end{equation}


%\section*{Вывод}

%<что сделали в конструкторке и получили в результате, кратко>



\clearpage
