\chapter{Конструкторская часть}


\section{Полный перебор}

Схема алгоритма полным перебором решения задачи коммивояжера представлена на рисунке~\ref{fig:std}.

\begin{figure}[h]
	\centering
	\includegraphics[scale=0.55]{/Users/rf9/PycharmProjects/aa/lab_6/report/images/bruteforce}
	\caption{Схема алгоритма полным перебором}
	\label{fig:std}
\end{figure}

\newpage

\section{Муравьиный алгоритм}

Схема муравьиного алгоритма решения задачи коммивояжера представлена на рисунке~\ref{fig:ant}. 
Схема алгоритма вычисления вероятности выбора следующей точки маршрута представлена на рисунке~\ref{fig:prob}.
Схема алгоритма обновления феромонов представлена на рисунке~\ref{fig:pher}.

\begin{figure}[h]
	\centering
	\includegraphics[scale=0.4]{/Users/rf9/PycharmProjects/aa/lab_6/report/images/ant}
	\caption{Схема муравьиного алгоритма}
	\label{fig:ant}
\end{figure}

\newpage

\begin{figure}[h]
	\centering
	\includegraphics[scale=0.35]{/Users/rf9/PycharmProjects/aa/lab_6/report/images/probabilities}
	\caption{Вычисление вероятности выбора следующей точки маршрута}
	\label{fig:prob}
\end{figure}

\newpage

\begin{figure}[h]
	\centering
	\includegraphics[scale=0.35]{/Users/rf9/PycharmProjects/aa/lab_6/report/images/pheromone}
	\caption{Обновление феромонов}
	\label{fig:pher}
\end{figure}


\section*{Вывод}

В данной части работы были описаны алгоритм полного перебора и муравьиный алгоритм решения задачи коммивояжера.
\clearpage
