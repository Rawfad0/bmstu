\section*{Задание}

Используя хвостовую рекурсию, разработать программу, позволяющую найти
\begin{enumerate}
\item n!,
\item п-е число Фибоначчи.
\end{enumerate}

Убедиться в правильности результатов.

Для одного из вариантов ВОПРОСА и каждого задания составить таблицу, отражающую конкретный порядок работы системы:

Т.к. резольвента хранится в виде стека, то состояние резольвенты требуется отображать в столбик: вершина - сверху! Новый шаг надо начинать с нового состояния резольвенты!

Для одного из вариантов ВОПРОСА и конкретной БЗ составить таблицу, отражающую конкретный порядок работы системы, с объяснениями: очередная проблема на каждом шаге и метод ее решения; каково новое текущее состояние резольвенты, как получено; какие дальнейшие действия? (Запускается ли алгоритм унификации? Каких термов? Почему этих?) ; вывод по результатам очередного шага и дальнейшие действия.

Содержание отчета

В отчете по лабораторной работе должны быть приведены:
\begin{enumerate}
	\item полный текст задания;
	\item текст программы;
	\item варианты вопросов;
	\item таблица, демонстрирующая работу системы при одном из успешных вариантов вопроса.

\clearpage
