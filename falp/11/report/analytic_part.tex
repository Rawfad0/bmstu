\section*{Задание}

Используя хвостовую рекурсию, разработать (комментируя назначение аргумента) программу, позволяющую:
\begin{enumerate}
\item Найти длину списка (по верхнему уровню);
\item Найти сумму элементов числового списка;
\item Найти сумму элементов числового списка, стоящих на нечетных позициях исходного списка (нумерация от 0);
\item Сформировать список из элементов числового списка, больших заданного значения;
\item Удалить заданный элемент из списка (один или все вхождения);
\item Объединить два списка.
\end{enumerate}

Убедиться в правильности результатов.

Для одного из вариантов ВОПРОСА уметь составить таблицу, отражающую конкретный порядок работы системы:

(Т.к. резольвента хранится в виде стека, то состояние резольвенты требуется отображать в столбик: вершина - сверху! Новый шаг надо начинать с нового состояния резольвенты!)

Содержание отчета

В отчете по лабораторной работе должны быть приведены:
\begin{enumerate}
	\item полный текст задания;
	\item текст программы;
	\item варианты вопросов;
	\item таблица, демонстрирующая работу системы при одном из успешных вариантов вопроса.
\end{enumerate}
\clearpage
