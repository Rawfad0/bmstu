\chapter{Исследовательская часть}


\section{Технические характеристики}

Характеристики устройства, на котором выполнялись замеры:

\begin{enumerate}
	\item операционная система --- macOS Sonoma 14.1 (23B2073);
	\item процессор --- Apple M3;
	\item оперативная память --- 16 Гб.
\end{enumerate}

\section{Время выполнения реализаций алгоритмов}

Замеры времени работы реализаций алгоритмов для каждого размера матриц проводились 100 раз, значение времени усреднялось. Данные замеров представлены в таблице~\ref{t:res}.


\begin{table}[h]
	\begin{center}
		\small
		\begin{threeparttable}
			\caption{Таблица замеров времени работы алгоритмов}
			\label{t:res}
			\begin{tabular}{|r@{\hspace{7mm}}|r@{\hspace{7mm}}|r@{\hspace{7mm}}|}
				\hline
				Размер & Полным перебором & Муравьиный алгоритм \\
				\hline
				2 & 0.00011 & 0.00011 \\ \hline
				3 & 0.00027 & 0.00027 \\ \hline
				4 & 0.00104 & 0.00104 \\ \hline
				5 & 0.00581 & 0.00581 \\ \hline
				6 & 0.03724 & 0.03724 \\ \hline
				7 & 0.27912 & 0.27912 \\ \hline
				8 & 2.41039 & 2.41039 \\ \hline
				9 & 23.09638 & 23.09638 \\ \hline
				10 & 243.90210 & 243.90210 \\ \hline
			\end{tabular}
		\end{threeparttable}
	\end{center}
\end{table}

\newpage

На рисунке~\ref{fig:fig1} показаны графики зависимости времени работы реализаций алгоритмов решения задачи коммивояжера от размера матриц.


\begin{figure}[h]
	\centering
	\includesvg[scale=0.65]{Figure_1.svg}
	\caption{Графики сравнения выполнения реализаций алгоритмов по времени}
	\label{fig:fig1}
\end{figure}

\section{Класс данных}

В качестве классов данных для параметризации используются три графа, построенных на городах Африки. В качестве стоимостей ребер использовались расстояния между городами по прямой.

\section{Результаты параметризации}

\begin{longtable}{|>{\centering\arraybackslash}r| >{\centering\arraybackslash}r| >{\centering\arraybackslash}r| >{\centering\arraybackslash}r| >{\centering\arraybackslash}r| >{\centering\arraybackslash}r| >{\centering\arraybackslash}r| >{\centering\arraybackslash}r| >{\centering\arraybackslash}r| >{\centering\arraybackslash}r| >{\centering\arraybackslash}r| >{\centering\arraybackslash}r|}
	\caption{Результаты параметризации муравьиного алгоритма}\label{tbl:param}
	\\ \hline
	\multicolumn{3}{|c|}{Параметры} & \multicolumn{3}{|c|}{Граф 1} & \multicolumn{3}{|c|}{Граф 2} & \multicolumn{3}{|c|}{Граф 3} \\
	\hline
	$\alpha$ & $\rho$ & \text{Дни} & \text{min} & \text{max} & \text{avg} & \text{min} & \text{max} & \text{avg} & \text{min} & \text{max} & \text{avg} \\
	\hline
	\endfirsthead
	\hline
	\endhead
	\hline
	\endfoot
	\endlastfoot
	\hline
	 0.1 & 0.3 & 400 & 38  & 38  & 38  & 5  & 5  & 5  & 101  & 101  & 101 \\
	0.1 & 0.3 & 500 & 38  & 38  & 38  & 5  & 5  & 5  & 101  & 101  & 101 \\
	0.1 & 0.4 & 400 & 38  & 38  & 38  & 5  & 5  & 5  & 101  & 101  & 101 \\
	0.1 & 0.4 & 500 & 38  & 38  & 38  & 5  & 5  & 5  & 101  & 101  & 101 \\
	0.1 & 0.5 & 400 & 38  & 38  & 38  & 5  & 5  & 5  & 101  & 101  & 101 \\
	0.1 & 0.6 & 100 & 38  & 38  & 38  & 5  & 5  & 5  & 101  & 101  & 101 \\
	0.1 & 0.7 & 500 & 38  & 38  & 38  & 5  & 5  & 5  & 101  & 101  & 101 \\
	0.1 & 0.8 & 300 & 38  & 38  & 38  & 5  & 5  & 5  & 101  & 101  & 101 \\
	0.2 & 0.3 & 400 & 38  & 38  & 38  & 5  & 5  & 5  & 101  & 101  & 101 \\
	0.2 & 0.3 & 500 & 38  & 38  & 38  & 5  & 5  & 5  & 101  & 101  & 101 \\
	\hline
\end{longtable}



\section{Вывод}

В результате исследования было получено, что на графе с количеством вершин менее 8 муравьиный алгоритм и алгоритм полного перебора решают задачу за примерно одинаковое время, а при количестве вершин большем или равном 8  алгоритм полного перебора начинает проигрывать в скорости муравьиному алгоритму.


\clearpage
