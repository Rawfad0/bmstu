\chapter{Конструкторская часть}


\section{Стандартный алгоритм}

Схема стандартного алгоритма умножения матриц представлена на рисунке~\ref{fig:std}.

\begin{figure}[h]
	\centering
	\includesvg[scale=0.7]{/Users/rf9/PycharmProjects/aa/lab_3/report/images/01.svg}
	\caption{Схема стандартного алгоритма умножения матриц}
	\label{fig:std}
\end{figure}

\newpage

\section{Алгоритм Винограда}

Схема алгоритма Винограда умножения матриц представлена на рисунках~\ref{fig:win} и~\ref{fig:win_2}.

\begin{figure}[h]
	\centering
	\includesvg[scale=0.8]{/Users/rf9/PycharmProjects/aa/lab_3/report/images/02.svg}
	\caption{Схема алгоритма Винограда умножения матриц}
	\label{fig:win}
\end{figure}

\newpage

\begin{figure}[h]
	\centering
	\includesvg[scale=0.9]{/Users/rf9/PycharmProjects/aa/lab_3/report/images/02_2.svg}
	\caption{Схема алгоритма Винограда умножения матриц}
	\label{fig:win_2}
\end{figure}

\newpage

\section{Оптимизированный алгоритм Винограда}

Схема оптимизированного алгоритма Винограда умножения матриц представлена на рисунках~\ref{fig:opt} и~\ref{fig:opt_2}.

\begin{figure}[h]
	\centering
	\includesvg[scale=0.9]{/Users/rf9/PycharmProjects/aa/lab_3/report/images/03.svg}
	\caption{Схема оптимизированного алгоритма Винограда умножения матриц}
	\label{fig:opt}
\end{figure}

\newpage

\begin{figure}[h]
	\centering
	\includesvg[scale=0.9]{/Users/rf9/PycharmProjects/aa/lab_3/report/images/03_2.svg}
	\caption{Схема оптимизированного алгоритма Винограда умножения матриц}
	\label{fig:opt_2}
\end{figure}

\newpage

\section{Трудоемкость алгоритмов}
Операции $+, -, +=, -=, []$ имеют трудоемкость 1.

Операции $*, /, \%$ имеют трудоемкость 2.

Трудоемкость условного оператора определяется формулой~\ref{eq:if}
\begin{equation}
	\label{eq:if}
	f_{\text{усл.оп.}} = 
	f_{\text{условия}} +
	\begin{cases}
		f_A, & \text{если условие выполняется,}\\
		f_B, & \text{иначе.}
	\end{cases}
\end{equation}

Трудоемкость цикла определяется формулой~\ref{eq:for}.
\begin{equation}
	\label{eq:for}
	f_{\text{цикл}} = f_{\text{инициализации}} + f_{\text{сравнения}} + N\cdot(f_{\text{тела}} + f_{\text{инкремента}} + f_{\text{сравнения}})
\end{equation}

\subsection{Трудоемкость стандартного алгоритма}
Стандартный алгоритм состоит из трех циклов.
Третий по вложенности цикл имеет трудоемкость $2 + 11 \cdot N$. Тогда второй по вложенности имеет трудоемкость $2 + K \cdot (2 + (2 + 11 \cdot N))$. Тогда внешний цикл имеет трудоемкость $f = 2 + M \cdot (2 + (2 + K \cdot (2 + (2 + 11 \cdot N))))$.
Т.~e. трудоемкость всего алгоритма равна $f_{\text{std}} = 2 + 4 \cdot M +  4 \cdot M \cdot K + 11 \cdot M \cdot K  \cdot N$.

\subsection{Трудоемкость алгоритма Винограда}
Трудоемкость заполнения вспомогательного массива $row\_factor$ равна $2 + M \cdot (6 + \frac{N}{2} \cdot 17)$.
Трудоемкость заполнения вспомогательного массива $col\_factor$ равна $4 + \frac{N}{2} \cdot (6 + K \cdot 15)$.
Трудоемкость наиболее вложенного цикла равна $4 + \frac{N}{2} \cdot 27$.
Трудоемкость основного цикла равна $2 + M \cdot (2 + 2 + K \cdot (2 + 7 + (4 + \frac{N}{2} \cdot 29)))$.
Трудоемкость условного оператора равна~\ref{eq:win_odd}.
\begin{equation}
	\label{eq:win_odd}
	2 +
	\begin{cases}
		0, & \text{размер четный,}\\
		2 + M \cdot (2 + 13 K), & \text{иначе.}
	\end{cases}
\end{equation}

Т.~e. трудоемкость всего алгоритма равна~\ref{eq:win}.
\begin{equation}
	\label{eq:win}
\begin{split}
	f_{\text{Винограда}} = (2 + 6 M + 8.5 \cdot M \cdot N) + (4 + 3 N + 7.5 \cdot N \cdot K) + \\
	+ (2 + 4 M + 13 \cdot M \cdot K + 14.5 \cdot M \cdot K \cdot N) + (2 + 
	\begin{cases}
		0, & \text{N четно}\\
		2 + 2 M +  13 \cdot M \cdot K, & \text{иначе}
	\end{cases}
	) = \\ =
	\begin{cases}
		10 + 10 M + 3 N + 8.5 \cdot M \cdot N + 7.5 \cdot N \cdot K + 13 \cdot M \cdot K + 14.5 \cdot M \cdot K \cdot N, & \text{л.с.}\\
		12 + 12  M + 3 N + 8.5 \cdot M \cdot N + 7.5 \cdot N \cdot K + 26 \cdot M \cdot K + 14.5 \cdot M \cdot K \cdot N, &\text{х. с.}
	\end{cases}
\end{split}
\end{equation}

\subsection{Трудоемкость оптимизированного алгоритма винограда}
Трудоемкость заполнения вспомогательного массива $row\_factor$ равна $2+ M\cdot (4+ \frac{N}{2}\cdot11)$.
Трудоемкость заполнения вспомогательного массива $col\_factor$ равна $2 + \frac{N}{2} \cdot (4 + K \cdot 11)$.
Трудоемкость вычисления флага для условия равна $3$.
Трудоемкость наиболее вложенного цикла равна $2 + \frac{N}{2} \cdot 19$.
Трудоемкость условного оператора равна~\ref{eq:opt_odd}.
\begin{equation}
	\label{eq:opt_odd}
	1 +
	\begin{cases}
		0, & \text{размер четный,}\\
		11, & \text{иначе.}
	\end{cases}
\end{equation}

Трудоемкость основного цикла равна~\ref{eq:main_for}.
\begin{equation}
	\label{eq:main_for}
	2 + M \cdot (2 + 2 + K \cdot (2 + 6 + (2 + \frac{N}{2} \cdot 19) + 1 + 
	\begin{cases}
			0, & \text{размер четный,}\\
			11, & \text{иначе.}
	\end{cases}))
\end{equation}

Т.~e. трудоемкость всего алгоритма равна~\ref{eq:opt}.
\begin{equation}
	\label{eq:opt}
	\begin{split}
		f_{\text{Опт. Винограда}} = (2 + 4 M + 5.5 \cdot M \cdot N) + (2 + 2 N + 5.5 \cdot N \cdot K) + \\
		+ (2 + 4 M + 11 \cdot M \cdot K + 9.5 \cdot M \cdot K \cdot N + \begin{cases}
			0, & \text{N четно}\\
			11 \cdot M \cdot K, & \text{иначе}
		\end{cases} = \\ =
		\begin{cases}
			6 + 8 M + 2 N + 5.5 \cdot M \cdot N + 5.5 \cdot N \cdot K + 11 \cdot M \cdot K + 9.5 \cdot M \cdot K \cdot N, & \text{л.с.}\\
			6 + 8 M + 2 N + 5.5 \cdot M \cdot N + 5.5 \cdot N \cdot K + 22 \cdot M \cdot K + 9.5 \cdot M \cdot K \cdot N, &\text{х. с.}
		\end{cases}
	\end{split}
\end{equation}

\section*{Вывод}

В данной части работы были описаны стандартный алгоритм, алгоритм Винограда, оптимизированный алгоритм Винограда умножения матриц. Оценены их трудоемкости.
\clearpage
