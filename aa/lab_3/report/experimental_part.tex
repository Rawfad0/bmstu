\chapter{Исследовательская часть}


\section{Технические характеристики}

Характеристики устройства, на котором выполнялись замеры:

\begin{enumerate}
	\item операционная система --- macOS Sonoma 14.1 (23B2073);
	\item процессор --- Apple M3;
	\item оперативная память --- 16 Гб.
\end{enumerate}

\section{Типы данных}
Матрица --- $matrix\_t$;
Тип ошибок --- $matrix\_error\_t$.

\section{Время выполнения реализаций алгоритмов}

Замеры времени работы реализаций алгоритмов для каждого размера матриц проводились 100 раз, значение времени усреднялось. Данные замеров представлены в таблице~\ref{t:res}.


\begin{table}[h]
	\begin{center}
		\small
		\begin{threeparttable}
			\caption{Таблица замеров времени работы алгоритмов (в мс)}
			\label{t:res}
			{\footnotesize
			\begin{tabular}{|r@{\hspace{7mm}}|r@{\hspace{7mm}}|r@{\hspace{7mm}}|r@{\hspace{7mm}}|}
				\hline
				Размер & Стандартный & Винограда & Оптимизированный Винограда \\
				\hline
				10 & 0.00934 & 0.00978 & 0.00626 \\
				\hline
				20 & 0.04994 & 0.03692 & 0.02672 \\
				\hline
				30 & 0.11589 & 0.07702 & 0.05016 \\
				\hline
				40 & 0.21056 & 0.12809 & 0.11678 \\
				\hline
				50 & 0.34473 & 0.22543 & 0.20771 \\
				\hline
				60 & 0.58648 & 0.37270 & 0.35031 \\
				\hline
				70 & 0.93209 & 0.59764 & 0.54972 \\
				\hline
				80 & 1.37477 & 0.92047 & 0.82010 \\
				\hline
				90 & 1.94044 & 1.32322 & 1.19331 \\
				\hline
				100 & 2.65325 & 1.76405 & 1.64636 \\
				\hline
				110 & 3.53173 & 2.36100 & 2.16105 \\
				\hline
				120 & 4.54108 & 2.99517 & 2.79426 \\
				\hline
				130 & 5.91079 & 3.87216 & 3.53079 \\
				\hline
				140 & 7.36001 & 4.76333 & 4.34239 \\
				\hline
				150 & 9.18645 & 5.93133 & 5.41438 \\
				\hline
				160 & 10.95094 & 7.06727 & 6.52982 \\
				\hline
				170 & 13.37995 & 8.55917 & 7.91273 \\
				\hline
				180 & 15.92595 & 9.97375 & 9.30066 \\
				\hline
				190 & 18.92719 & 11.87942 & 10.79745 \\
				\hline
				200 & 21.91092 & 13.64643 & 12.61243 \\
				\hline
				210 & 25.83687 & 15.94249 & 14.57796 \\
				\hline
				220 & 29.91168 & 18.10955 & 16.70312 \\
				\hline
				230 & 34.30373 & 20.82617 & 19.24681 \\
				\hline
				240 & 38.81086 & 23.48579 & 21.71082 \\
				\hline
				250 & 47.14721 & 26.66110 & 24.61888 \\
				\hline
				260 & 50.02556 & 29.69702 & 27.84935 \\
				\hline
				270 & 59.25041 & 33.38113 & 31.85158 \\
				\hline
				280 & 62.57422 & 37.20468 & 35.80270 \\
				\hline
				290 & 69.20933 & 41.30339 & 40.40267 \\
				\hline
				300 & 80.53249 & 45.57732 & 44.66144 \\
				\hline
			\end{tabular}}
		\end{threeparttable}
	\end{center}
\end{table}

\newpage

На рисунке~\ref{fig:fig1} показаны графики зависимости времени работы реализаций алгоритмов умножения матриц от размера матриц. На рисунке~\ref{fig:fig2} показаны графики зависимости времени работы реализаций алгоритмов умножения матриц методом Винограда от размера матриц.


\begin{figure}[h]
	\centering
	\includesvg[scale=0.65]{Figure_1.svg}
	\caption{Графики сравнения выполнения реализаций алгоритмов по времени}
	\label{fig:fig1}
\end{figure}

\begin{figure}[h]
	\centering
	\includesvg[scale=0.65]{Figure_2.svg}
	\caption{Графики сравнения выполнения реализаций алгоритмов метода Винограда по времени}
	\label{fig:fig2}
\end{figure}

\section{Вывод}

В данном разделе были приведены технические характеристики и время выполнения реализаций алгоритмов.
Стандартный алгоритм умножения матриц выполнялся дольше других алгоритмов. Неоптимизированный алгоритм Винограда выполнялся дольше оптимизированного.


\clearpage
