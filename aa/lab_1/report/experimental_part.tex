\chapter{Исследовательская часть}


\section{Технические характеристики}

Характеристики устройства (STM32F767~\cite{lit4}), на котором выполнялись замеры времени:

\begin{enumerate}
	\item процессор	 Arm® 32-bit Cortex®-M7 CPU, 216 МГц;
	\item оперативная память:  512 Кбайт.
\end{enumerate}

\section{Время выполнения алгоритмов}

Замеры времени работы алгоритмов для каждой длины строк проводились по 100 раз для каждого алгоритма (строки случайно сгенерируются на каждой итерации, т.е. на итерации все алгоритмы работают с одинаковыми строками), затем значение времени усреднялось. Данные представлены в таблице ~\ref{t:time}.

\begin{table}[!ht]
	\begin{center}
		\small
		\begin{threeparttable}
			\caption{Таблица замеров времени работы алгоритмов (в мс)}
			\label{t:time}
			\begin{tabular}{|c|r|r|r|r|}
				\hline
				\bfseries Длина строк 
				& \bfseries  Левенштейн
				& \bfseries  Лев. рек.
				& \bfseries  Лев. рек. с кэшем
				& \bfseries  Дамерау~---~Левенштейн \\
				\hline
				0 & 0.09 & 0.01 & 0.12 & 0.11 \\
				\hline
				1 & 0.15 & 0.06 & 0.14 & 0.14 \\
				\hline
				2 & 0.20 & 0.23 & 0.28 & 0.24 \\
				\hline
				3 & 0.35 & 1.15 & 0.57 & 0.40 \\
				\hline
				4 & 0.58 & 5.95 & 0.94 & 0.70 \\
				\hline
				5 & 0.82 & 30.97 & 1.39 & 1.03 \\
				\hline
			\end{tabular}	
		\end{threeparttable}
	\end{center}
\end{table}
На рисунке~\ref{fig:compare} показан график зависимости времени работы алгоритмов нахождения расстояния Левенштейна и Дамерау~---~Левенштейна от длины строк. 

\begin{figure}[ht]
	\centering
	\includesvg[scale=0.7]{Figure_1.svg}
	\caption{График сравнения алгоритмов на строках до длины 4}
	\label{fig:compare}
\end{figure}

\newpage

На рисунке~\ref{fig:compare_without_rec} показан график зависимости времени работы тех же алгоритмов, что и на рисунке~\ref{fig:compare}, но без учета рекурсивного алгоритма без кэширования.

\begin{figure}[ht]
	\centering
	\includesvg[scale=0.6]{Figure_2.svg}
	\caption{График сравнения алгоритмов (кроме рекурсивного без кэша)}
	\label{fig:compare_without_rec}
\end{figure}

На рисунке~[\ref{fig:compare_nonrec}] показан график зависимости времени работы нерекурсивных алгоритмов от длины строк.

\begin{figure}[ht]
	\centering
	\includesvg[scale=0.6]{Figure_3.svg}
	\caption{График сравнения нерекурсивных алгоритмов нахождения расстояний Левенштейна и Дамерау~---~Левенштейна}
	\label{fig:compare_nonrec}
\end{figure}


\section{Вывод}

В данном разделе были приведены технические характеристики и время выполнения алгоритмов.
Рекурсивный алгоритм без кэширования, вычисляющий расстояние Левенштейна, выполнялся дольше других вариантов. Нерекурсивный алгоритм вычисления расстояния Дамерау~---~Левенштейна выполнялся медленнее, чем нерекурсивный алгоритм вычисления расстояния Левенштейна, т.к. содержал дополнительные действия.

\clearpage
