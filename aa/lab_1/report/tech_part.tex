\chapter{Технологическая часть}

\section{Требования к программному обеспечению}

Входные данные: две строки на русском или английском языке в любом регистре;

Выходные данные: искомое расстояние для каждого метода.

\section{Средства реализации}
Алгоритмы для данной лабораторной работы были реализованы на языке MicroPython (реализация Python) с помощью функции замера времени ticks\_ms~\cite{lit3}.

\section{Реализация алгоритмов}

Реализация алгоритма нахождения расстояния Левенштейна и Дамерау~---~Левенштейна представлена в листинга~(\ref{lst:lev_dist})~-~ ~(\ref{lst:dam_lev_dist}).

\begin{center}
	\captionsetup{justification=raggedright,singlelinecheck=off}
	\begin{lstlisting}[label=lst:lev_dist,caption=Функция нахождения расстояния Левенштейна нерекурсивно]
def lev_dist(s1, s2):
	l1, l2 = len(s1), len(s2)
	matrix = [[0 for _ in range(l2 + 1)] for _ in range(l1 + 1)]
	
	for i in range(l1 + 1):
		matrix[i][0] = i
	for j in range(l2 + 1):
		matrix[0][j] = j
		
	for i in range(1, l1 + 1):
		for j in range(1, l2 + 1):
			change = 1 if s1[i - 1] != s2[j - 1] else 0
			matrix[i][j] = min(
				matrix[i][j - 1] + 1,
				matrix[i - 1][j] + 1,
				matrix[i - 1][j - 1] + change
			)
	return matrix[l1][l2]
	\end{lstlisting}
\end{center}

\begin{center}
	\captionsetup{justification=raggedright,singlelinecheck=off}
	\begin{lstlisting}[label=lst:lev_dist_rec,caption=Функция нахождения расстояния Левенштейна рекурсивно]
def lev_dist_rec(s1, s2, l1, l2):
	if l1 == 0:
		return l2
		
	if l2 == 0:
		return l1
	
	change = 1 if s1[l1 - 1] != s2[l2 - 1] else 0
	res = min(
		lev_dist_rec(s1, s2, l1, l2 - 1) + 1,
		lev_dist_rec(s1, s2, l1 - 1, l2) + 1,
		lev_dist_rec(s1, s2, l1 - 1, l2 - 1) + change
	)
	return res
	\end{lstlisting}
\end{center}

\begin{center}
	\captionsetup{justification=raggedright,singlelinecheck=off}
	\begin{lstlisting}[label=lst:lev_dist_rec_cache,caption=Функция нахождения расстояния Левенштейна рекурсивно с кэшем]
def lev_rec_cache_entry(s1, s2):
	l1, l2 = len(s1), len(s2)
	
	matrix = [[-1 for _ in range(l2 + 1)] for _ in range(l1 + 1)]
	lev_rec_cache(s1, s2, l1, l2, matrix)
	return matrix[l1][l2]

def lev_rec_cache(s1, s2, l1, l2, matrix):
	if l1 == 0:
		matrix[l1][l2] = l2
	elif l2 == 0:
		matrix[l1][l2] = l1
	else:
		if matrix[l1 - 1][l2] == -1:
			lev_rec_cache(s1, s2, l1 - 1, l2, matrix)
		if matrix[l1][l2 - 1] == -1:
			lev_rec_cache(s1, s2, l1, l2 - 1, matrix)
		if matrix[l1 - 1][l2 - 1] == -1:
			lev_rec_cache(s1, s2, l1 - 1, l2 - 1, matrix)
		
		change = 1 if s1[l1 - 1] != s2[l2 - 1] else 0
		matrix[l1][l2] = min(
			matrix[l1][l2 - 1] + 1,
			matrix[l1 - 1][l2] + 1,
			matrix[l1 - 1][l2 - 1] + change
		)
	\end{lstlisting}
\end{center}

\newpage

\begin{center}
	\captionsetup{justification=raggedright,singlelinecheck=off}
	\begin{lstlisting}[label=lst:dam_lev_dist,caption=Функция нахождения расстояния Дамерау~---~Левенштейна нерекурсивно]
def dam_lev_dist(s1, s2):
	l1, l2 = len(s1), len(s2)
	matrix = [[0 for _ in range(l2 + 1)] for _ in range(l1 + 1)]
	
	for i in range(l1 + 1):
		matrix[i][0] = i
	for j in range(l2 + 1):
		matrix[0][j] = j
	
	for i in range(1, l1 + 1):
		for j in range(1, l2 + 1):
			change = 1 if s1[i - 1] != s2[j - 1] else 0
			matrix[i][j] = min(
				matrix[i][j - 1] + 1,
				matrix[i - 1][j] + 1,
				matrix[i - 1][j - 1] + change
			)
			if i > 1 and j > 1 and s1[i - 1] == s2[j - 2] and s1[i - 2] == s2[j - 1]:
				matrix[i][j] = min(matrix[i][j], matrix[i - 2][j - 2] + 1)
	return matrix[l1][l2]
	\end{lstlisting}
\end{center}

\section{Тестирование}

В таблице~[\ref{t:tests}] представлены данные о результатах тестирования реализации алгоритмов нахождения расстояния Левенштейна и Дамерау~---~Левенштейна.
Все тесты пройдены успешно.

\begin{table}[!ht]
	
	\begin{center}
		\small
		\begin{threeparttable}
			\caption{Модульные тесты}
			\label{t:tests}
			\begin{tabular}{|c|c|c|c|c|c|}
				\hline
				\multicolumn{2}{|c|}{\bfseries Входные данные}
				& \multicolumn{4}{c|}{\bfseries Алгоритм и расстояние} \\ 
				\hline 
				&
				& \multicolumn{3}{c|}{\bfseries Левенштейна} 
				& \multicolumn{1}{c|}{\bfseries Дамерау~---~Левенштейна} \\  
				\cline{3-6}
				
				\bfseries Строка 1 
				& \bfseries Строка 2 
				& \bfseries Итеративный 
				& \multicolumn{2}{c|}{\bfseries Рекурсивный} 
				& \bfseries Итеративный \\ 
				\cline{4-5}
				
			 	& & & \bfseries Без кэша & \bfseries С кэшем & \\
				\hline
				polynomial & exponential & 6 & 6 & 6 & 6 \\
				\hline
				silent & listen & 4 & 4 & 4 & 4 \\
				\hline
				кот & скат & 2 & 2 & 2 & 2 \\
				\hline
				ab & ba & 2 & 2 & 2 & 1 \\
				\hline
			\end{tabular}	
		\end{threeparttable}
	\end{center}
\end{table}

\section*{Вывод}

В данном разделе были рассмотрены требования к программному обеспечению, используемые средства реализации, приведена реализация алгоритмов и результаты тестирования.

\clearpage
