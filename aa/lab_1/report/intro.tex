\ssr{ВВЕДЕНИЕ}

Расстояние Левенштейна (англ. Levenshtein distance) (также редакционное расстояние или дистанция редактирования) между двумя строками в теории информации и компьютерной лингвистике~\cite{lit1} --- это минимальное количество операций вставки одного символа, удаления одного символа и замены одного символа на другой (также транспозиции двух соседних символов для расстояния Дамерау~---~Левенштейна~\cite{lit2}), необходимых для превращения одной строки в другую.

\textbf{Цель лабораторной работы} --- сравнение алгоритмов нахождения расстояния Левенштейна и Дамерау~---~Левенштейна. Для достижения поставленной цели необходимо выполнить следующие задачи:
\begin{enumerate}
	\item реализовать указанные алгоритмы поиска расстояния (два матричных, рекурсивный без кэширования и рекурсивный с кэшированием);
	\item провести замеры затраченного процессорного времени выполнения реализованных алгоритмов;
	\item провести исследование затрачиваемого процессорного времени и памяти при различных реализациях алгоритмов
	\item провести сравнительный анализ реализаций алгоритмов по полученным  данным;
	\item описать результаты в отчете.
\end{enumerate}

\clearpage
