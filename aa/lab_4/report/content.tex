% Содержимое отчета по курсу Анализ алгоритмов

\aaunnumberedsection{ВВЕДЕНИЕ}{sec:intro}

Цель работы --- сравнение алгоритмов последовательной и параллельной обработки на основе нативных потоков.

Для достижения поставленной цели необходимо выполнить следующие задачи: 
\begin{itemize}
    \item реализовать алгоритмы последовательной и параллельной обработки данных;
    \item провести замеры времени выполнения реализаций алгоритмов;
    \item провести сравнительный анализ реализаций алгоритмов по полученным данным;
    \item описать результаты в отчете.
\end{itemize}

\aasection{Входные и выходные данные}{sec:input-output}

Входными данными программы являются адрес главной страницы ресурса, максимальное количество страниц, с которых выгружаются данные, и количество потоков.
Выходными данными является директория с файлами, которые содержат скачанные данные со страниц в текстовом формате.

\aasection{Тестирование}{sec:tests}

В таблице~\ref{tbl:tests} представлены функциональные тесты для разработанного ПО. Все тесты пройдены успешно.


\begin{longtable}{|p{.1\textwidth - 2\tabcolsep}|p{.33\textwidth - 2\tabcolsep}|p{.24\textwidth - 2\tabcolsep}|p{.23\textwidth - 2\tabcolsep}|}
    \caption{Функциональные тесты}\label{tbl:tests} \\\hline
    № & Входные данные & Ожидаемые выходные данные & Результат тестирования                                          \\\hline
    \endfirsthead
    \endfoot
    1                                           & https://www.patee.ru 20 1 & Директория recipes и скачанные рецепты & Тест пройден \\\hline
    2                                           & https://www.patee.ru 20 8 & Директория recipes и скачанные рецепты & Тест пройден \\\hline
    3                                           & https://www.patee.org & ERROR & Тест пройден \\\hline
    \end{longtable}

\aasection{Описание исследования}{sec:study}

Необходимо исследовать зависимость производительности разработанного ПО от количества потоков. Изменять количество потоков от 1 до $4 \cdot 16$.

Технические характеристики устройства, на котором выполнялись замеры: 
\begin{enumerate}
	\item операционная система --- macOS Sonoma 14.1 (23B2073);
	\item процессор --- Apple M3 (количество ядер: 14)~\cite{1};
	\item оперативная память --- 16 Гб.
\end{enumerate}

В таблице~\ref{tbl:1} приведено усредненное время за 5 обработок по 100 страниц в миллисекундах в зависимости от количества потоков. В таблице~\ref{tbl:2} приведена производительность обработки в страницах в секунду в зависимости от количества потоков.

%\begin{longtable}{|p{\hspace{7mm}}|p{\hspace{7mm}|}
%\caption{Время обработки 100 страниц в миллисекундах)}
\begin{table}[h]
   	\begin{center}
   		\small
   		\begin{threeparttable}
   			\caption{Время обработки 100 страниц в миллисекундах)}
    		\label{tbl:1}
    		\begin{tabular}{|r@{\hspace{7mm}}|r@{\hspace{7mm}}|}
			    \hline 
			    Количество потоков & Время работы \\ 
			    \hline
			    1   & 127353.0     \\ 
			    \hline
			    2   &  87003.0     \\ 
			    \hline
			    4   &  59079.6     \\ 
			    \hline
			    8   & 41243.2     \\ 
			    \hline
			    16   &  24837.0    \\ 
			    \hline
			    32   & 15494.4     \\ 
			    \hline
			    64   & 8386.8    \\ 
			    \hline
			\end{tabular}
		\end{threeparttable}
	\end{center}
\end{table}


\begin{table}[h]
	\begin{center}
		\small
		\begin{threeparttable}
			\caption{Количество обработанных страниц в секунду}
			\label{tbl:2}
			\begin{tabular}{|r@{\hspace{7mm}}|r@{\hspace{7mm}}|}
				\hline 
				Количество потоков & Производительность \\ 
				\hline
				1 & 0.785 \\
				\hline
				2 & 1.149 \\
				\hline
				4 & 1.693 \\
				\hline
				8 & 2.425 \\
				\hline
				16 & 4.026 \\
				\hline
				32 & 6.454 \\
				\hline
				64 & 11.923 \\
				\hline
			\end{tabular}
		\end{threeparttable}
	\end{center}
\end{table}


\newpage

%\end{longtable}

На рисунке~\ref{fig:fig1} показан график зависимости времени работы программы от количества потоков. На рисунке~\ref{fig:fig2} показан график зависимости производительности разработанного ПО от количества потоков.

\begin{figure}[h!]
	\centering
	\includesvg[scale=0.8]{inc/img/Figure_1.svg}
	\caption{Зависимость времени выполнения программы от количества потоков}
	\label{fig:fig1}
\end{figure}

\newpage

\begin{figure}[h!]
	\centering
	\includesvg[scale=0.8]{inc/img/Figure_2.svg}
	\caption{Зависимость производительности программы от количества потоков}
	\label{fig:fig2}
\end{figure}


Параллельная обработка заняла меньше времени, чем последовательная. При заданном количестве страниц, производительность растет при увеличении количества потоков.
% линейно зависит от количества потоков. 


\newpage

\aaunnumberedsection{ЗАКЛЮЧЕНИЕ}{sec:outro}

Цель работы достигнута: сравнение алгоритмов последовательной и параллельной обработки на основе нативных потоков было проведено. 
В ходе выполнения лабораторной работы были решены все задачи: 
\begin{itemize}
	\item реализованы алгоритмы последовательной и параллельной обработки данных;
	\item проведены замеры времени выполнения реализаций алгоритмов;
	\item проведен сравнительный анализ реализаций алгоритмов по полученным данным;
	\item описаны результаты в отчете.
\end{itemize}
