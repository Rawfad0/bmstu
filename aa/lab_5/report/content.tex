% Содержимое отчета по курсу Анализ алгоритмов

\aaunnumberedsection{ВВЕДЕНИЕ}{sec:intro}

Цель работы --- получение навыка организации параллельных вычислений по конвейерному принципу.

Для достижения поставленной цели необходимо выполнить следующие задачи: 
\begin{itemize}
	\item определить входные и выходные данные;
	\item реализовать алгоритмы параллельных вычислений по конвейерному принципу;
	\item протестировать программу;
    \item провести анализ логированных данных;
    \item описать результаты в отчете.
\end{itemize}

\aasection{Входные и выходные данные}{sec:input-output}

Входными данными программы является адрес директории с рецептами.
Выходными данными являются файл с логированными данными и строки в таблице рецептов базы данных.

\aasection{Тестирование}{sec:tests}

В таблице~\ref{tbl:tests} представлены функциональные тесты для разработанного ПО. Все тесты пройдены успешно.


\begin{longtable}{|p{.1\textwidth - 2\tabcolsep}|p{.33\textwidth - 2\tabcolsep}|p{.24\textwidth - 2\tabcolsep}|p{.23\textwidth - 2\tabcolsep}|}
    \caption{Функциональные тесты}\label{tbl:tests} \\\hline
    № & Входные данные & Ожидаемые выходные данные & Результат тестирования                                          \\\hline
    \endfirsthead
    \endfoot
    1                                           & ./recipes & log.txt и строки в базе данных & Тест пройден \\\hline
    2                                           & ./htmls & ERROR & Тест пройден \\\hline
    3                                           & ./notexist & ERROR & Тест пройден \\\hline
    \end{longtable}

\aasection{Описание исследования}{sec:study}

Необходимо проанализировать файлы $log.txt$ и $log_res.txt$, в которых находится время добавления и удаления из очереди, начала и окончания выполнения стадий обработки задач.

Технические характеристики устройства, на котором выполнялись замеры: 
\begin{enumerate}
	\item операционная система --- macOS Sonoma 14.1 (23B2073);
	\item процессор --- Apple M3 (количество ядер: 14)~\cite{1};
	\item оперативная память --- 16 Гб.
\end{enumerate}

% В таблице~\ref{tbl:1} приведено усредненное время за 5 обработок по 100 страниц в миллисекундах в зависимости от количества потоков. В таблице~\ref{tbl:2} приведена производительность обработки в страницах в секунду в зависимости от количества потоков.

В таблице~\ref{tbl:1} приведены результаты логирования первых 7 задач в хронологическом порядке из файла $log.txt$.


\newpage

%\begin{longtable}{|p{\hspace{7mm}}|p{\hspace{7mm}|}
%\caption{Время обработки 100 страниц в миллисекундах)}
\begin{table}[h]
   	\begin{center}
   		\small
   		\begin{threeparttable}
   			\caption{Содержимое файла $log.txt$}
    		\label{tbl:1}
    		\begin{tabular}{|r@{\hspace{7mm}}|l@{\hspace{7mm}}|}
			    \hline 
			    Временная метка & Событие \\ 
			    \hline
11:30:29.238 & 1: чтение --- начало\\ \hline
11:30:29.239 & 1: чтение --- окончание\\ \hline
11:30:29.239 & 2: чтение --- начало\\ \hline
11:30:29.239 & 1: парсинг --- начало\\ \hline
11:30:29.239 & 2: чтение --- окончание\\ \hline
11:30:29.239 & 3: чтение --- начало\\ \hline
11:30:29.239 & 3: чтение --- окончание\\ \hline
11:30:29.239 & 4: чтение --- начало\\ \hline
11:30:29.239 & 4: чтение --- окончание\\ \hline
11:30:29.240 & 1: парсинг --- окончание\\ \hline
11:30:29.240 & 2: парсинг --- начало\\ \hline
11:30:29.240 & 2: парсинг --- окончание\\ \hline
11:30:29.240 & 3: парсинг --- начало\\ \hline
11:30:29.240 & 3: парсинг --- окончание\\ \hline
11:30:29.240 & 4: парсинг --- начало\\ \hline
11:30:29.240 & 4: парсинг --- окончание\\ \hline
11:30:29.241 & 1: запись --- начало\\ \hline
11:30:29.241 & 1: запись --- окончание\\ \hline
11:30:29.241 & 2: запись --- начало\\ \hline
11:30:29.241 & 2: запись --- окончание\\ \hline
11:30:29.241 & 3: запись --- начало\\ \hline
11:30:29.241 & 3: запись --- окончание\\ \hline
11:30:29.241 & 4: запись --- начало\\ \hline
11:30:29.241 & 4: запись --- начало\\ \hline
11:30:29.241 & 4: запись --- окончание\\ \hline
			\end{tabular}
		\end{threeparttable}
	\end{center}
\end{table}

В результате анализа файла логирования было подтверждено, что обработка данных по конвейерному принципу выполняется параллельно. 

%В таблице~\ref{tbl:2} приведены результаты конвейерной обработки $962$ рецептов.
Среднее время ожидания задачи равны $0.084$, $0.105$, $0.197$ мс для трех очередей .
Среднее время обработки задачи равно $0.011$, $0.196$, $0.202$ мс для трех обработчиков.



\newpage

\newpage

\aaunnumberedsection{ЗАКЛЮЧЕНИЕ}{sec:outro}

Цель работы достигнута: получен навык организации параллельных вычислений по конвейерному принципу.
В ходе выполнения лабораторной работы были решены все задачи: 
\begin{itemize}
	\item определены входные и выходные данные;
	\item реализованы алгоритмы параллельных вычислений по конвейерному принципу;
	\item протестирована программу;
	\item проведен анализ логированных данных;
	\item описаны результаты в отчете.
\end{itemize}
