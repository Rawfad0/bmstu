\chapter{Аналитическая часть}

В этом разделе будет представлена информация о задаче коммивояжера, а также о способах её решения --- методом полного перебора и методом на основе муравьиного алгоритма.

\section{Задача коммивояжера}

Задача о коммивояжере \text{(англ. \textit{Traveling Salesman Problem}), TSP} --- задача, в которой коммивояжер должен посетить $N$ городов, побывав в каждом из них ровно по одному разу и завершив путешествие в том городе, с которого он начал. В какой последовательности ему нужно обходить города, чтобы общая длина его пути была наименьшей?~\cite{lit1}

\section{Полный перебор}

Можно решить задачу перебором всевозможных перестановок. Для этого нужно сгенерировать все $N!$ всевозможных перестановок вершин исходного графа, подсчитать для каждой перестановки длину маршрута и выбрать минимальный из них. Но тогда задача оказывается неосуществимой даже для достаточно небольших $N$~\cite{lit1}.

\section{Муравьиный алгоритм}

В основе муравьиного алгоритма лежит идея моделирования поведения колонии муравьев. Каждый муравей определяет свой маршрут на основе оставленных другими муравьями феромонов, а также сам оставляет оставляет феромоны, чтобы последующие муравьи ориентировались по ним. В результате при прохождении каждым муравьем своего маршрута наибольшее число феромонов остается на самом оптимальном пути. Временная сложность алгоритма была оценена как $683 - (42,467 · N) + (1,0696 · N^2)$~\cite{lit2}. Однако главный недостаток алгоритма заключается в том, что, по сравнению с алгоритмом полного перебора, он дает приближенное решение задачи, а не точное.


\section*{Вывод}

В данном разделе рассмотрены методы решения задачи коммивояжера.

\clearpage
