\chapter{Технологическая часть}

\section{Требования к программному обеспечению}

Входные данные: массив и искомое значение;

Выходные данные: позиция искомого значения в массиве и количество сравнений.

\section{Средства реализации}
Алгоритмы для данной лабораторной работы были реализованы на языке $Python$~\cite{lit3}, который имеет библиотеку для визуализации данных $Matplotlib$~\cite{lit4}.

\section{Реализация алгоритмов}

Реализация алгоритмов нахождения искомого значения полным перебором и двоичным поиском представлена в листингах~(\ref{lst:lin_search}) и~(\ref{lst:bin_search}).

\begin{center}
	\captionsetup{justification=raggedright,singlelinecheck=off}
	\begin{lstlisting}[label=lst:lin_search,caption=Функция нахождения искомого значения полным перебором]
def linear_search(arr, x):
	comparisons, xi = 0, -1
	for i in range(len(arr)):
		comparisons += 1
		if arr[i] == x:
			xi = i
			break
	return xi, comparisons
	\end{lstlisting}
\end{center}

\begin{center}
	\captionsetup{justification=raggedright,singlelinecheck=off}
	\begin{lstlisting}[label=lst:bin_search,caption=Функция нахождения искомого значения двоичным поиском]
def binary_search(arr, x):
	comparisons, xi = 0, -1
	l, r = 0, len(arr) - 1
	while l <= r:
		m = (l + r) // 2
		comparisons += 1
		if arr[m] < x:
			l = m + 1
		elif arr[m] > x:
			r = m - 1
		else:
			xi = m
			break
	return xi, comparisons
	\end{lstlisting}
\end{center}

\section{Тестирование}

В таблице~\ref{t:tests} представлены данные о результатах тестирования реализации алгоритмов нахождения искомого значения полным перебором и двоичным поиском.

\begin{table}[!ht]
	
	\begin{center}
		\small
		\begin{threeparttable}
			\caption{Модульные тесты}
			\label{t:tests}
			\begin{tabular}{|c|c|r|r|r|r|}
				\hline
				\multicolumn{2}{|c|}{\bfseries Входные данные}
				& \multicolumn{4}{c|}{\bfseries Алгоритм, индекс и количество сравнений} \\ 
				\hline 
				\bfseries Массив
				& \bfseries Искомое значение
				& \multicolumn{2}{c|}{\bfseries Полный перебор}
				& \multicolumn{2}{c|}{\bfseries Двоичный поиск} \\  
				\hline
				arr \{...\} & 34772 & 454 & 455 & 331 & 9 \\
				\hline
				arr \{...\} & 34773 & -1 & 1033 & -1 & 10 \\
				\hline
				arr \{...\} & 999 & -1 & 1033 & -1 & 10 \\
				\hline
			\end{tabular}	
		\end{threeparttable}
	\end{center}
\end{table}
Все тесты пройдены успешно.

\section*{Вывод}

В данном разделе были рассмотрены требования к программному обеспечению, используемые средства реализации, приведена реализация алгоритмов и результаты тестирования.

\clearpage
