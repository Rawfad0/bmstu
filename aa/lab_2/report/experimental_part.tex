\chapter{Исследовательская часть}


\section{Технические характеристики}

Характеристики устройства, на котором выполнялись замеры:

\begin{enumerate}
	\item операционная система --- macOS Sonoma 14.1 (23B2073);
	\item процессор --- Apple M3;
	\item оперативная память --- 16 Гб.
\end{enumerate}

\section{Количество сравнений}


Согласно варианту задания, массив состоит из $N=1033$ элементов.
Оценка трудоемкости алгоритма измерена в количестве сравнений, потребовавшихся для поиска искомого значения в массиве в каждом из $N+1$ случаев ($N$ искомых значений, существующих в массиве, и одно искомое значение, которого нет в массиве).

На рисунке~\ref{fig:linear} показана гистограмма зависимости количества сравнений от индекса искомого значения в массиве при использовании алгоритма нахождения искомого значения полным перебором. 

\begin{figure}[h]
	\centering
	\includesvg[scale=0.65]{Figure_1.svg}
	\caption{Гистограмма для алгоритма полным перебором}
	\label{fig:linear}
\end{figure}

\newpage

На рисунке~\ref{fig:binary} показана гистограмма зависимости количества сравнений от индекса искомого значения в массиве при использовании алгоритма нахождения искомого значения двоичным поиском. 

\begin{figure}[h]
	\centering
	\includesvg[scale=0.65]{Figure_2.svg}
	\caption{Гистограмма для алгоритма двоичным поиском}
	\label{fig:binary}
\end{figure}

На рисунке~\ref{fig:binary_sorted} показана гистограмма количества сравнений при использовании алгоритма нахождения искомого значения двоичным поиском, отсортированная по возрастанию количества сравнений

\begin{figure}[h]
	\centering
	\includesvg[scale=0.65]{Figure_3.svg}
	\caption{Гистограмма для алгоритма двоичным поиском отсортированная по возрастанию количества сравнений}
	\label{fig:binary_sorted}
\end{figure}


\section{Вывод}

В данном разделе были приведены технические характеристики и время выполнения алгоритмов.
Количество сравнений в алгоритме полным перебором зависит от индекса искомого элемента и может достигать $N$ в худших случаях. В алгоритме двоичного поиска количество сравнений в худших случаях равно $\lceil \log_2 N \rceil$, что для  $N=1033$ равно $11$. Это означает, что алгоритм двоичного поиска в большинстве случаев оказывался быстрее алгоритма полным перебором. Недостатком алгоритма двоичного поиска является необходимость в сортировке массива перед поиском. 

%, но если поиск по данному массиву проводится многократно, то

\clearpage
