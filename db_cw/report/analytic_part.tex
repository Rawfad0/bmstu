\chapter{Аналитический раздел}

В этом разделе будет проведен анализ предметоной области, сравнительный анализ существующих решений и анализ существующих баз данных. Будет формализована задача и требования к разрабатываемой базе данных и приложению. Будет проведена формализация данных и ролей в базе данных и приложении.

\section{Анализ предметной области}

При планировании покупки некоторого товара, пользователи перебирают множество вариантов товара, просматривают множество сайтов магазинов, чтобы найти самое выгодное предложение того самого товара. 

Однако можно не тратить время на перебор всех вариантов и комбинаций продавцов-товаров, а воспользоваться маркетплейсом, где все можно найти все товары от различных продавцов, выбрать необходимый, а затем получить список предложений из разных магазинов. Все удобно собрано в одном месте.

\clearpage

\section{Анализ существующих решений}

На российском рынке выделяются несколько крупных маркетплейсов: \textit{Wildberries}~\cite{lit6}, \textit{Ozon}~\cite{lit7}, \textit{Яндекс Маркет}~\cite{lit8} и \textit{МегаМаркет}~\cite{lit9}. Несмотря на общую цель --- предоставить пользователю возможность найти и купить товар, решения различаются по функциональным возможностям.

\if 0
\textbf{Ozon} является одной из крупнейших торговых онлайн-площадок России. Система обеспечивает удобный интерфейс для покупателей, развитую систему фильтрации и персональных рекомендаций. Для продавцов предусмотрены аналитика продаж, управление ценами и остатками на складах. Основная особенность базы данных --- необходимость обработки огромного объёма транзакций в реальном времени и поддержка динамического изменения цен.

\textbf{Wildberries} занимает лидирующую позицию на российском рынке. Основной упор сделан на широту ассортимента и удобную логистику. Для обеспечения высокой скорости обслуживания используется интеграция базы данных с внутренними складскими системами. Особое внимание уделяется информации о наличии товаров, возвратах и отзывах. Ключевой проблемой является сложность работы с интерфейсом и ограниченные возможности аналитики для продавцов.

\textbf{Яндекс Маркет} ориентирован не только на покупателей, но и на формирование экосистемы сервисов. Помимо информации о товарах и заказах, база данных интегрируется с системой контекстной рекламы, отзывами и рейтинговыми системами. Отличительной особенностью является развитая система рекомендаций, основанная на анализе пользовательских данных и поведенческих факторов.

\textbf{СберМегаМаркет} (ранее Goods.ru) активно развивается благодаря интеграции в экосистему Сбербанка. Особенность заключается в глубокой интеграции с финансовыми сервисами и удобных платёжных инструментах. В структуре базы данных особое внимание уделяется транзакциям и надёжности хранения платёжной информации. Недостатком является сравнительно меньший ассортимент по сравнению с Ozon и Wildberries.
\fi

Для анализа выбраны следующие ключевые критерии:
\begin{itemize}
	\item \textbf{Поиск товаров (группировка по товарам)} --- пользователь видит список уникальных товаров, а все предложения от разных продавцов скрыты внутри карточки товара.
	\item \textbf{Поиск предложений (вывод всех предложений сразу)} --- пользователь видит сразу все доступные предложения конкретного товара от разных продавцов, без группировки по товарам.
	\item \textbf{Добавление в избранное} --- возможность сохранять товары или предложения для последующего просмотра.
	\item \textbf{Написание отзывов} --- возможность оставлять оценки и комментарии к товарам или предложениям.
	\item \textbf{Рекомендации} --- система персональных или тематических рекомендаций товаров на основе истории просмотров или покупок.
\end{itemize}

Результаты сравнения представлены в таблице~\ref{tab:compare_simple}.

\begin{table}[h!]
	\centering
	\caption{Сравнение российских маркетплейсов по ключевым функциям}
	\label{tab:compare_simple}
	\begin{tabular}{|p{5cm}|c|c|c|c|}
		\hline
		\textbf{Критерий} & \textbf{Wildberries} & \textbf{Ozon} & \textbf{Яндекс Маркет} & \textbf{МегаМаркет} \\ \hline
		Поиск товаров (группировка по товарам) & Нет & Нет & Есть & Нет \\ \hline
		Поиск предложений (вывод всех предложений сразу) & Есть & Есть & Есть & Есть \\ \hline
		Добавление в избранное & Есть & Есть & Есть & Есть \\ \hline
		Написание отзывов & Есть & Есть & Есть & Есть \\ \hline
		Рекомендации & Есть & Есть & Есть & Есть \\ \hline
	\end{tabular}
\end{table}

Таким образом, даже при схожем базовом функционале различия проявляются в способе отображения товаров и предложений.

\clearpage

\section{Формализация задачи}

Необходимо спроектировать и разработать приложение и базу данных для взаимодействия с товарами и предложениями. 

Покупатель должен иметь возможность просматривать товары, предложения, свои заказы, отзывы, избранные товары.

Продавец должен иметь возможность просматривать, добавлять и изменять свои предложения и магазины. 

Администратор должен иметь возможность добавлять товары и изменять информацию о товарах, предложениях и магазинах.

\clearpage

\section{Формализация данных}

База данных для маркетплейса должна содержать информацию о покупателях, продавцах, магазинах, товарах, предложениях, заказах, отзывах и избранных. 

На рисунке~\ref{fig:er_diagram} приведена диаграмма сущность-связь базы данных маркетплейса в нотации Чена.

\begin{figure}[h!]
	\centering
	\includesvg[scale=0.65]{images/er.drawio.svg}
	\caption{Диаграмма сущность-связь базы данных маркетплейса в нотации Чена}
	\label{fig:er_diagram}
\end{figure}

Данная диаграмма описывает сущности базы данных, их атрибуты и связи между этими сущностями.

\clearpage

\section{Формализация ролей}

Для взаимодействия с приложением маркетплейса было выделено четыре роли пользователя: гость, покупатель, продавец и администратор.

\textbf{Гость} (неавторизованный пользователь) является начальным состоянием пользователя в приложении. Гость имеет возможность зарегистрироваться, то есть создать покупателя или продавца, и войти в качестве покупателя, продавца или администратора. 

На рисунке~\ref{fig:use_case_guest} представлена диаграмма вариантов использования для роли гость.

\begin{figure}[h!]
	\centering
	\includesvg[scale=0.8]{images/use_case_guest.drawio.svg}
	\caption{Диаграмма вариантов использования для роли Гость}
	\label{fig:use_case_guest}
\end{figure}

\textbf{Покупатель} имеет возможность просматривать товары, предложения, свои заказы, отзывы, избранные товары, добавлять приложение в заказ, оплачивать заказы, добавлять и удалять товар из изранного, писать и удалять отзывы.

На рисунке~\ref{fig:use_case_buyer} представлена диаграмма вариантов использования для роли покупатель.

\begin{figure}[h!]
	\centering
	\includesvg[scale=0.75]{images/use_case_customer.drawio.svg}
	\caption{Диаграмма вариантов использования для роли Покупатель}
	\label{fig:use_case_buyer}
\end{figure}

\textbf{Продавец} имеет возможность просматривать, добавлять и изменять свои предложения и магазины. 

На рисунке~\ref{fig:use_case_seller} представлена диаграмма вариантов использования для роли продавец.

\begin{figure}[h!]
	\centering
	\includesvg[scale=0.8]{images/use_case_seller.drawio.svg}
	\caption{Диаграмма вариантов использования для роли Продавец}
	\label{fig:use_case_seller}
\end{figure}

\textbf{Администратор} имеет возможность добавлять товары и изменять информацию о товарах, предложениях и магазинах.

На рисунке~\ref{fig:use_case_admin} представлена диаграмма вариантов использования для роли администратор.

\begin{figure}[h!]
	\centering
	\includesvg[scale=0.8]{images/use_case_admin.drawio.svg}
	\caption{Диаграмма вариантов использования для роли Администратор}
	\label{fig:use_case_admin}
\end{figure}

\clearpage

\section{Анализ существующих баз данных}

\subsection{Иерархическая модель}

Иерархическая модель описывает данные в форме иерархических древовидных структур, где каждая иерархия представляет собой совокупность связанных записей. Унифицированного стандарта языка для этой модели не существует. Наиболее известным языком манипулирования данными является DL/1, используемый в системе IMS. Данная система занимала лидирующее положение на рынке СУБД более 20 лет (с 1965 по 1985 гг.), а её язык DL/1 в течение длительного времени выступал фактическим промышленным стандартом~\cite{lit1}.

\subsection{Сетевая модель}

Сетевая модель представляет данные в виде типов записей и поддерживает связь «один-ко-многим» (set type) с помощью указателей. Известна как модель CODASYL DBTG и использует язык обработки данных «запись за записью», встроенный в язык программирования-хост. Язык манипулирования данными (DML) для сетевой модели предложен в DBTG Report (1971) как расширение COBOL~\cite{lit1}.

% Fundamentals of Database Systems Seventh Edition

\subsection{Реляционная модель}

Реляционная модель представляет базу данных как совокупность отношений и основывается на следующих условиях~\cite{lit3}:

\begin{itemize}
	\item \textbf{Структурный аспект.} Данные в базе воспринимаются пользователем исключительно как таблицы.
	\item \textbf{Аспект целостности.} Эти таблицы удовлетворяют определённым условиям целостности (которые будут рассмотрены в конце раздела).
	\item \textbf{Аспект обработки.} В распоряжении пользователя имеются операторы манипулирования таблицами (например, предназначенные для поиска данных), которые генерируют новые таблицы на основании уже имеющихся и среди которых есть, по крайней мере, операторы \textit{сокращения} (restrict), \textit{проекции} (project) и \textit{объединения} (join).
\end{itemize}

% введение в базы данных Дейт

\clearpage

\subsection{Постреляционная модель}

% Под постреляционными понимают системы, предлагающие более общий подход к моделированию данных, чем реляционная модель. Встречаются и другие термины: гибридные базы данных, объектно-расширенные СУБД и др. В отличие от реляционной модели, такие решения не ограничены Принципом информации Э. Ф. Кодда и допускают хранение данных не только в виде отношений. Некоторые расширения заимствуют идеи дореляционных технологий, например, поддержку графов (как в GraphDB компании sones). Другие комбинируют реляционные и нереляционные возможности, что позволяет даже исторически дореляционным системам (PICK, MUMPS) претендовать на постреляционный статус. Примером альтернативного подхода является модель ресурсного пространства (RSM), основанная на многомерной классификации.

Постреляционные модели — системы с более общим подходом к данным, чем реляционная модель. Они не ограничены принципом информации Кодда и допускают хранение данных не только в виде отношений. Некоторые расширяют реляционные системы поддержкой графов (например, GraphDB~\cite{lit10}), другие комбинируют реляционные и дореляционные возможности (PICK, MUMPS). Альтернативный подход — модель ресурсного пространства (RSM) на основе многомерной классификации.

\subsection{Анализ моделей баз данных}

Для проведения анализа сформулируем следующие критерии, выполнение которых необходимо для выполнения поставленых задач:

\begin{enumerate}
	\item Данные описываются стандартизированным языком запросов;
	\item Есть универсальная модель связей;
	\item Логика не зависит от физического хранения.
\end{enumerate}

Результаты сравнения моделей баз данных указаны в таблице~\ref{tab:db_models_comparison2}.

\begin{table}[h!]
	\centering
	\small
	\caption{Сравнение моделей баз данных по ключевым критериям}
	\label{tab:db_models_comparison2}
	\begin{tabular}{|l|c|c|c|}
		\hline
		\textbf{Модель БД} & 
		\textbf{Критерий 1} & 
		\textbf{Критерий 2} & 
		\textbf{Критерий 3} \\
		\hline
		Иерархическая & Нет & Нет & Нет \\
		\hline
		Сетевая & Нет & Частично & Нет \\
		\hline
		Реляционная & Да & Да & Да \\
		\hline
		Постреляционная & Частично & Да & Да \\
		\hline
	\end{tabular}
\end{table}

В ходе сравнения была выбрана база данных с реляционной моделью, так как она удовлетворяет перечисленным критериям.

\clearpage

\section*{Вывод}

В аналитическом разделе был проведен анализ предметоной области, сравнительный анализ существующих решений и анализ существующих баз данных. Была формализована задача и требования к разрабатываемой базе данных и приложению. Была проведена формализация данных и ролей в базе данных и приложении.

\clearpage
