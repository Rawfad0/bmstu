\chapter{Конструкторский раздел}

В данном разделе будет проведено описание сущностей базы данных, описание атрибутов, типов данных и ограничений, проектирование базы данных, описание ролевой модели.
%В данном разделе будут спроектированы базы данных, описаны сущности, роли и триггеры. 

\section{Описание сущностей базы данных}

В таблицах~\ref{tbl:customer}-\ref{tbl:review} приведено описание сущностей баз данных.


\begin{table}[h]
	\centering
	\caption{Описание сущности «Покупатель»}
	\label{tbl:customer}
	\begin{tabularx}{\textwidth}{|l|l|l|X|}
		\hline
		\textbf{Атрибут} & \textbf{Значение} & \textbf{Тип данных} & \textbf{Описание} \\ \hline
		Идентификатор & id & UUID & Первичный ключ \\ \hline
		Фамилия & surname & Строковый & Обязателен; не пустая строка \\ \hline
		Имя & name & Строковый & Обязателен; не пустая строка \\ \hline
		Номер телефона & phone & Строковый & Обязателен; уникален; формат телефонного номера \\ \hline
		Эл. почта & email & Строковый & Обязателен; уникален; корректный формат email \\ \hline
		Дата рождения & birthdate & Дата & Может быть пустым \\ \hline
		Дата регистрации & registration\_date & Дата & Обязателен; по умолчанию — текущая дата \\ \hline
		Бонусные баллы & bonus & Целочисленный & По умолчанию 0 \\ \hline
	\end{tabularx}
\end{table}

\begin{table}[h]
	\centering
	\caption{Описание сущности «Продавец»}
	\label{tbl:seller}
	\begin{tabularx}{\textwidth}{|l|l|l|X|}
		\hline
		\textbf{Атрибут} & \textbf{Значение} & \textbf{Тип данных} & \textbf{Описание} \\ \hline
		Идентификатор & id & UUID & Первичный ключ \\ \hline
		Фамилия & surname & Строковый & Обязателен; не пустая строка \\ \hline
		Имя & name & Строковый & Обязателен; не пустая строка \\ \hline
		Номер телефона & phone & Строковый & Уникален; формат телефонного номера \\ \hline
		Эл. почта & email & Строковый & Уникален; корректный формат email \\ \hline
		Дата рождения & birthdate & Дата & Может быть пустым \\ \hline
	\end{tabularx}
\end{table}

\begin{table}[h]
	\centering
	\caption{Описание сущности «Магазин»}
	\label{tbl:store}
	\begin{tabularx}{\textwidth}{|l|l|l|X|}
		\hline
		\textbf{Атрибут} & \textbf{Значение} & \textbf{Тип данных} & \textbf{Описание} \\ \hline
		Идентификатор & id & UUID & Первичный ключ \\ \hline
		Наименование & name & Строковый & Обязателен; уникален; не пустая строка \\ \hline
		ОГРН & ogrn & Строковый & Уникален; обязательное поле \\ \hline
		ИП & ip & Строковый & Может быть пустым \\ \hline
		Количество оценок & rating\_count & Целочисленный & По умолчанию 0 \\ \hline
		Средняя оценка & rating\_avg & Числовой (дробный) & По умолчанию 0.0; диапазон 0–5 \\ \hline
	\end{tabularx}
\end{table}

\begin{table}[h]
	\centering
	\caption{Описание сущности «Товар»}
	\label{tbl:product}
	\begin{tabularx}{\textwidth}{|l|l|l|X|}
		\hline
		\textbf{Атрибут} & \textbf{Значение} & \textbf{Тип данных} & \textbf{Описание} \\ \hline
		Идентификатор & id & UUID & Первичный ключ \\ \hline
		Наименование & name & Строковый & Обязателен; не пустая строка \\ \hline
		Категория & category & Строковый & Обязателен \\ \hline
		Описание & description & Текстовый & Может быть пустым \\ \hline
		Количество оценок & rating\_count & Целочисленный & По умолчанию 0 \\ \hline
		Средняя оценка & rating\_avg & Числовой (дробный) & По умолчанию 0.0; диапазон 0–5 \\ \hline
	\end{tabularx}
\end{table}

\begin{table}[h]
	\centering
	\caption{Описание сущности «Предложение»}
	\label{tbl:offer}
	\begin{tabularx}{\textwidth}{|l|l|l|X|}
		\hline
		\textbf{Атрибут} & \textbf{Значение} & \textbf{Тип данных} & \textbf{Описание} \\ \hline
		Идентификатор & id & UUID & Первичный ключ \\ \hline
		ID товара & product\_id & UUID & Внешний ключ на «Товар» \\ \hline
		ID магазина & store\_id & UUID & Внешний ключ на «Магазин» \\ \hline
		Цена & price & Числовой (дробный) & Обязателен; >0 \\ \hline
		Количество & quantity & Целочисленный & Обязателен; >=0 \\ \hline
		Срок доставки & delivery\_time & Целочисленный & Количество дней; может быть пустым \\ \hline
	\end{tabularx}
\end{table}

\begin{table}[h]
	\centering
	\caption{Описание сущности «Заказ»}
	\label{tbl:order}
	\begin{tabularx}{\textwidth}{|l|l|l|X|}
		\hline
		\textbf{Атрибут} & \textbf{Значение} & \textbf{Тип данных} & \textbf{Описание} \\ \hline
		Идентификатор & id & UUID & Первичный ключ \\ \hline
		ID предложения & offer\_id & UUID & Внешний ключ на «Предложение» \\ \hline
		Количество & quantity & Целочисленный & Обязателен; >0 \\ \hline
	\end{tabularx}
\end{table}

\begin{table}[h]
	\centering
	\caption{Описание сущности «Избранное»}
	\label{tbl:favorite}
	\begin{tabularx}{\textwidth}{|l|l|l|X|}
		\hline
		\textbf{Атрибут} & \textbf{Значение} & \textbf{Тип данных} & \textbf{Описание} \\ \hline
		Идентификатор & id & UUID & Первичный ключ \\ \hline
		ID покупателя & customer\_id & UUID & Внешний ключ на «Покупатель» \\ \hline
		ID товара & product\_id & UUID & Внешний ключ на «Товар» \\ \hline
	\end{tabularx}
\end{table}

\begin{table}[h]
	\centering
	\caption{Описание сущности «Отзыв»}
	\label{tbl:review}
	\begin{tabularx}{\textwidth}{|l|l|l|X|}
		\hline
		\textbf{Атрибут} & \textbf{Значение} & \textbf{Тип данных} & \textbf{Описание} \\ \hline
		Идентификатор & id & UUID & Первичный ключ \\ \hline
		ID покупателя & customer\_id & UUID & Внешний ключ на «Покупатель» \\ \hline
		ID товара & product\_id & UUID & Внешний ключ на «Товар» \\ \hline
		Оценка & rating & Числовой & Обязательна; диапазон 0–5 \\ \hline
		Текст отзыва & text & Текстовый & Может быть пустым; длина ограничена \\ \hline
	\end{tabularx}
\end{table}

\clearpage

\section{Диаграмма проектируемой базы данных}

На рисунке~\ref{fig:er_diagram2} приведена диаграмма сущность-связь проектируемой базы данных в нотации Чена.

\begin{figure}[h!]
	\centering
	\includesvg[scale=0.65]{images/er.drawio.svg}
	\caption{Диаграмма сущность-связь проектируемой базы данных в нотации Чена}
	\label{fig:er_diagram2}
\end{figure}

\clearpage

\section{Описание проектируемой ролевой модели на уровне БД}

Определены следующие роли:

\begin{enumerate}
	\item \textbf{Пользователь} (marketplace\_user). Роль позволяет просматривать данные таблиц товаров, предложений, магазинов. Внесение изменений недоступно.
	
	\item \textbf{Менеджер} (marketplace\_manager). Эта роль включает в себя все возможности пользователя. Дополнительно менеджер получает право на просмотр и редактирование всех таблиц базы данных, а также добавление товаров.
	
	\item \textbf{Администратор} (marketplace\_admin). Эта роль обладает полным доступом и не имеет ограничений.
\end{enumerate}

\clearpage

\section*{Вывод}

В данном разделе было проведено описание сущностей базы данных, описание атрибутов, типов данных и ограничений, проектирование базы данных, описание ролевой модели.

%<что сделали в конструкторке и получили в результате, кратко>

\clearpage
