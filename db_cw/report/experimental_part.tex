\chapter{Исследовательский раздел}

%<цель исследования, на какой машине делали (указать ЦПУ, ОЗУ, ОС), желательно написать, как исследовали и при каких условиях; что получили в результате (таблицы + графики)>


%<что сделали и получили в результате, кратко>

\section{Технические характеристики}

Характеристики устройства, на котором выполнялись замеры~\cite{lit21}:

\begin{enumerate}
	\item операционная система --- macOS Sonoma 14.1 (23B2073);
	\item процессор --- Apple M3;
	\item оперативная память --- 16 Гб.
\end{enumerate}

\clearpage

\section{Цель исследования}

Целью исследования является сравнение времени выполнения запроса \textit{GetOfferDetailsByProductId} с индексом~\textit{idx\_offers\_product\_id} и без него в зависимости от количества строк в таблице~\textit{offers}.

\clearpage

\section{Результаты исследования}

В таблице~\ref{tab:index_performance_small} представлены результаты сравнения врмени выполнения запроса с индексом и без него в зависимости от количества строк в таблице~\textit{offers}.

\begin{table}[H]
	\centering
	\caption{Сравнение времени выполнения запроса с индексом и без него}
	\label{tab:index_performance_small}
	\begin{tabular}{|c|c|c|}
		\hline
		\textbf{Количество строк} & \textbf{Без индекса (мс)} & \textbf{С индексом (мс)} \\ \hline
		100   & 1.2 & 1.1 \\ \hline
		200   & 2.3 & 1.9 \\ \hline
		300   & 3.5 & 2.7 \\ \hline
		400   & 4.8 & 3.5 \\ \hline
		500   & 6.1 & 4.2 \\ \hline
		600   & 7.5 & 4.9 \\ \hline
		700   & 8.9 & 5.4 \\ \hline
		800   & 10.3 & 6.0 \\ \hline
		900   & 11.8 & 6.6 \\ \hline
		1000  & 13.5 & 7.2 \\ \hline
	\end{tabular}
\end{table}

На рисунке~\ref{fig:figure1} представлены графики зависимости времени врмени выполнения запроса с индексом и без него в зависимости от количества строк в таблице~\textit{offers}.

\begin{figure}[h!]
	\centering
	\includesvg[scale=0.7]{images/Figure_1.svg}
	\caption{Графики зависимости времени выполнения запроса с индексом и без него от количества строк}
	\label{fig:figure1}
\end{figure}

\clearpage

\section*{Вывод}

Из проведенного исследования следует, что наличие индекса (по умолчанию \textit{B-tree}) подходящего под частый запрос (запрос предложений --- один из основных запросов маркетплейса) уменьшает время выполнения запроса относительно запроса без индекса (обычный последовательный поиск \textit{sequential scan}), тем самым ускоряя работу приложения и повышая отзывчивость для пользователя, что улучшает его пользовательский опыт.

\clearpage
