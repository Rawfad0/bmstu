\documentclass[12pt,a4paper,oneside]{report}
\usepackage[utf8]{inputenc}
\usepackage[english,russian]{babel}
\usepackage{amsmath}
\usepackage{amssymb}
\usepackage{geometry}
\usepackage{sverb}
\usepackage{graphicx}
\usepackage{pdfpages}
\usepackage{hyperref} 
\usepackage{url}
\usepackage{titlesec, blindtext, color}
\usepackage{listings}
\usepackage{pgfplots}
\pgfplotsset{compat=newest}
\graphicspath{{../}}
\DeclareGraphicsExtensions{.pdf,.png,.jpg}
\usepackage{tabularx}
\usepackage{subcaption}
\usepackage{colortbl}
\usepackage{multirow}
\usepackage{longtable}
\usepackage{enumitem}
\usepackage{algorithm}
\usepackage{tikz}
\usepackage[noend]{algpseudocode}
\usepackage{float}
\definecolor{gray75}{gray}{0.75}
\definecolor{Blue}{HTML}{5D8AA8}
\newcommand{\hsp}{\hspace{20pt}}

\newcommand{\RomanNumeralCaps}[1]
{\MakeUppercase{\romannumeral #1}}



% Для листинга кода:
\lstset{ %
	language=c,                 % выбор языка для подсветки (здесь это С)
	basicstyle=\small\sffamily, % размер и начертание шрифта для подсветки кода             % где поставить нумерацию строк (слева\справа)
	numberstyle=\tiny,           % размер шрифта для номеров строк
	stepnumber=1, 
	keywordstyle=\color{blue},% размер шага между двумя номерами строк
	numbersep=5pt,                % как далеко отстоят номера строк от подсвечиваемого кода
	showspaces=false,            % показывать или нет пробелы специальными отступами
	showstringspaces=false,      % показывать или нет пробелы в строках
	showtabs=false,             % показывать или нет табуляцию в строках
	frame=single,              % рисовать рамку вокруг кода
	tabsize=2,                 % размер табуляции по умолчанию равен 2 пробелам
	captionpos=t,              % позиция заголовка вверху [t] или внизу [b] 
	breaklines=true,           % автоматически переносить строки (да\нет)
	breakatwhitespace=false, % переносить строки только если есть пробел
	escapeinside={\#*}{*)}  % Стиль литералов
}


\titleformat{\chapter}[hang]{\Huge\bfseries}{\thechapter.\textcolor{gray75}\hsp}{0pt}{\Huge\bfseries}
\newcommand{\specchapter}[1]{\chapter*{#1}\addcontentsline{toc}{chapter}{#1}}
\newcommand{\specsection}[1]{\section*{#1}\addcontentsline{toc}{section}{#1}}
\newcommand{\specsubsection}[1]{\subsection*{#1}\addcontentsline{toc}{subsection}{#1}}

% геометрия


\setcounter{tocdepth}{4} % фикс переноса 
\righthyphenmin = 2
\tolerance = 2048


\thispagestyle{empty}

\geometry{pdftex, left = 3cm, right = 1cm, top = 2cm, bottom = 2cm}

\begin{document}
	
\thispagestyle{empty}


\noindent \begin{minipage}{0.15\textwidth}
	\includegraphics[width=\linewidth]{images/b_logo}
\end{minipage}
\noindent\begin{minipage}{0.85\textwidth}\centering
	\fontsize{12pt}{12pt}\selectfont
	\textbf{Министерство науки и высшего образования Российской Федерации}\\
	\textbf{Федеральное государственное автономное образовательное учреждение высшего образования}\\
	\textbf{«Московский государственный технический университет имени Н.Э.~Баумана}\\
	\textbf{(национальный исследовательский университет)»}\\
	\textbf{(МГТУ им. Н.Э.~Баумана)}
\end{minipage}

\noindent\rule{\linewidth}{3pt}
\newline\newline
\noindent ФАКУЛЬТЕТ $\underline{\text{«Информатика и системы управления»}}$ \newline\newline
\noindent КАФЕДРА $\underline{\text{«Программное обеспечение ЭВМ и информационные технологии»}}$

\vspace{1cm}

\begin{center}
	\noindent\begin{minipage}{1.3\textwidth}\centering
		\Large\textbf{Лабораторная работа № 3}\newline
		\textbf{по дисциплине <<Моделирование>>}\newline\newline
	\end{minipage}
\end{center}

\noindent\textbf{Тема} $\underline{\text{Распределения случайных величин}}$\newline\newline
\noindent\textbf{Студент} $\underline{\text{Равашдех Ф.Х.}}$\newline\newline
\noindent\textbf{Группа} $\underline{\text{ИУ7-75Б}}$\newline\newline
\noindent\textbf{Преподаватель} $\underline{\text{Рудаков И.В..}}$\newline

\begin{center}
	\vfill
	Москва,~\the\year
\end{center}
\clearpage



\section*{Задание}
\quad Разработать программное обеспечение, выводящее графики функций плотностей распределений и функций распределений для равномерного, Пуассоновского, экспоненциального, нормального, Эрланговского распределений
в зависимости от заданных параметров соответствующих распределений.


\section*{Теория}

\subsection*{1. Равномерное распределение $X \sim \text{U}[a,b]$}
\begin{align}
	\text{Функция плотности распределения:} \quad & f_X(x) =
	\begin{cases}
		\dfrac{1}{b-a}, & a \le x \le b, \\
		0, & \text{иначе.}
	\end{cases} \\
	\text{Функция распределения:} \quad & F_X(x) =
	\begin{cases}
		0, & x < a, \\
		\dfrac{x-a}{b-a}, & a \le x \le b, \\
		1, & x > b.
	\end{cases} \\
	M[X] = \frac{a+b}{2}, \quad & D[X] = \frac{(b-a)^2}{12}
\end{align}

\subsection*{2. Пуассоновское распределение $X \sim \text{P}(\lambda)$}
\begin{align}
	\text{Функция вероятности:} \quad & P(X=k) = \frac{\lambda^k e^{-\lambda}}{k!}, \quad k = 0,1,2,\dots \\
	\text{Функция распределения:} \quad & F_X(k) = \sum_{i=0}^{k} \frac{\lambda^i e^{-\lambda}}{i!}, \quad k = 0,1,2,\dots \\
	M[X] = \lambda, \quad & D[X] = \lambda
\end{align}

\subsection*{3. Экспоненциальное распределение $X \sim \text{Exp}(\lambda)$}
\begin{align}
	\text{Функция плотности распределения:} \quad & f_X(x) =
	\begin{cases}
		\lambda e^{-\lambda x}, & x \ge 0, \\
		0, & x < 0
	\end{cases} \\
	\text{Функция распределения:} \quad & F_X(x) =
	\begin{cases}
		0, & x < 0, \\
		1 - e^{-\lambda x}, & x \ge 0
	\end{cases} \\
	M[X] = \frac{1}{\lambda}, \quad & D[X] = \frac{1}{\lambda^2}
\end{align}

\subsection*{4. Нормальное распределение $X \sim N(\mu,\sigma^2)$}
\begin{align}
	\text{Функция плотности распределения:} \quad & f_X(x) = \frac{1}{\sigma \sqrt{2 \pi}} \exp \Bigg( - \frac{(x-\mu)^2}{2\sigma^2} \Bigg) \\
	\text{Функция распределения:} \quad & F_0(x) =  \frac{1}{\sqrt{2\pi}} \int_{-\infty}^{x} e^{\frac{-t^2}{2}} dt \\
	M[X] = \mu, \quad & D[X] = \sigma^2
\end{align}

\subsection*{5. k-Эрланговское распределение $X \sim \text{E}(k,\lambda)$}
\begin{align}
	\text{Функция плотности распределения:} \quad & f_X(x) = \frac{\lambda (\lambda x)^{k-1} e^{-\lambda x}}{(k-1)!}, \quad x \ge 0 \\
	\text{Функция распределения:} \quad & F_X(x) = 1 - \sum_{n=0}^{k-1} \frac{(\lambda x)^n e^{-\lambda x}}{n!}, \quad x \ge 0 \\
	M[X] = \frac{k}{\lambda}, \quad & D[X] = \frac{k}{\lambda^2}
\end{align}



\clearpage
	
\end{document}
