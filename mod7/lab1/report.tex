\documentclass[12pt,a4paper,oneside]{report}
\usepackage[utf8]{inputenc}
\usepackage[english,russian]{babel}
\usepackage{amsmath}
\usepackage{amssymb}
\usepackage{geometry}
\usepackage{sverb}
\usepackage{graphicx}
\usepackage{pdfpages}
\usepackage{hyperref} 
\usepackage{url}
\usepackage{titlesec, blindtext, color}
\usepackage{listings}
\usepackage{pgfplots}
\pgfplotsset{compat=newest}
\graphicspath{{../}}
\DeclareGraphicsExtensions{.pdf,.png,.jpg}
\usepackage{tabularx}
\usepackage{subcaption}
\usepackage{colortbl}
\usepackage{multirow}
\usepackage{longtable}
\usepackage{enumitem}
\usepackage{algorithm}
\usepackage{tikz}
\usepackage[noend]{algpseudocode}
\usepackage{float}
\definecolor{gray75}{gray}{0.75}
\definecolor{Blue}{HTML}{5D8AA8}
\newcommand{\hsp}{\hspace{20pt}}

\newcommand{\RomanNumeralCaps}[1]
{\MakeUppercase{\romannumeral #1}}



% Для листинга кода:
\lstset{ %
	language=c,                 % выбор языка для подсветки (здесь это С)
	basicstyle=\small\sffamily, % размер и начертание шрифта для подсветки кода             % где поставить нумерацию строк (слева\справа)
	numberstyle=\tiny,           % размер шрифта для номеров строк
	stepnumber=1, 
	keywordstyle=\color{blue},% размер шага между двумя номерами строк
	numbersep=5pt,                % как далеко отстоят номера строк от подсвечиваемого кода
	showspaces=false,            % показывать или нет пробелы специальными отступами
	showstringspaces=false,      % показывать или нет пробелы в строках
	showtabs=false,             % показывать или нет табуляцию в строках
	frame=single,              % рисовать рамку вокруг кода
	tabsize=2,                 % размер табуляции по умолчанию равен 2 пробелам
	captionpos=t,              % позиция заголовка вверху [t] или внизу [b] 
	breaklines=true,           % автоматически переносить строки (да\нет)
	breakatwhitespace=false, % переносить строки только если есть пробел
	escapeinside={\#*}{*)}  % Стиль литералов
}


\titleformat{\chapter}[hang]{\Huge\bfseries}{\thechapter.\textcolor{gray75}\hsp}{0pt}{\Huge\bfseries}
\newcommand{\specchapter}[1]{\chapter*{#1}\addcontentsline{toc}{chapter}{#1}}
\newcommand{\specsection}[1]{\section*{#1}\addcontentsline{toc}{section}{#1}}
\newcommand{\specsubsection}[1]{\subsection*{#1}\addcontentsline{toc}{subsection}{#1}}

% геометрия


\setcounter{tocdepth}{4} % фикс переноса 
\righthyphenmin = 2
\tolerance = 2048


\thispagestyle{empty}

\geometry{pdftex, left = 3cm, right = 1cm, top = 2cm, bottom = 2cm}

\begin{document}
	
	\thispagestyle{empty}
	
	
	\noindent \begin{minipage}{0.15\textwidth}
		\includegraphics[width=\linewidth]{images/b_logo}
	\end{minipage}
	\noindent\begin{minipage}{0.85\textwidth}\centering
		\fontsize{12pt}{12pt}\selectfont
		\textbf{Министерство науки и высшего образования Российской Федерации}\\
		\textbf{Федеральное государственное автономное образовательное учреждение высшего образования}\\
		\textbf{«Московский государственный технический университет имени Н.Э.~Баумана}\\
		\textbf{(национальный исследовательский университет)»}\\
		\textbf{(МГТУ им. Н.Э.~Баумана)}
	\end{minipage}
	
	\noindent\rule{\linewidth}{3pt}
	\newline\newline
	\noindent ФАКУЛЬТЕТ $\underline{\text{«Информатика и системы управления»}}$ \newline\newline
	\noindent КАФЕДРА $\underline{\text{«Программное обеспечение ЭВМ и информационные технологии»}}$
	
	\vspace{1cm}
	
	\begin{center}
		\noindent\begin{minipage}{1.3\textwidth}\centering
			\Large\textbf{Лабораторная работа № 1}\newline
			\textbf{по дисциплине <<Моделирование>>}\newline\newline
		\end{minipage}
	\end{center}
	
	\noindent\textbf{Тема} $\underline{\text{Оценка случайности сгенерированных последовательностей}}$\newline\newline
	\noindent\textbf{Студент} $\underline{\text{Равашдех Ф.Х.}}$\newline\newline
	\noindent\textbf{Группа} $\underline{\text{ИУ7-75Б}}$\newline\newline
	\noindent\textbf{Преподаватель} $\underline{\text{Рудаков И.В..}}$\newline
	
	\begin{center}
		\vfill
		Москва,~\the\year
	\end{center}
	\clearpage
	
	
	
	\section*{Задание}
	\quad Реализовать критерий оценки случайной последовательностии сравнить результаты работы данного критерия на одноразрядных, двухразрядных и трехразрядных последовательностях целых чисел. Последовательности получать табличным и алгоритмическим способами.
	
	\section*{Табличный метод}
	
	\quad В данной работе для генерации случайных чисел табличным способом используются цифры из части таблицы <<A~Million Random Digits with~100,000~Normal~Deviates>>, опубликованной в 1955 году. Данная таблица сохранена в виде текстового файла. Для генерации числа читается число из файла, затем отсекается необходимое количество разрядов. 
	
	\section*{Линейный конгруэнтный метод}
	\quad В данной работе был выбран линейный конгруэнтный метод в качестве алгоритмического.
	Для осуществления генерации чисел данным методом, необходимо задать 4 числа:
	
	\begin{center}
		m > 0, модуль\\
		0 $\leq$ a	$\leq$ m, \text{множитель}\\
		0 $\leq$ c $\leq$ m, \text{приращение}\\
		0 $\leq$ $X_0$ $\leq$ m, \text{начальное число}\\
	\end{center}
	
	Последовательность случайных чисел генерируется при помощи формулы:
	\begin{equation}
		X_{n+1}=(aX_n+c)\mod m
	\end{equation}
	
	\section*{Собственный критерий}
	
	\quad Используется комбинированный критерий, который использует три критерия для получения оценки случайности последовательности:
	\begin{enumerate}
		\item проверка на среднее арифметическое значение;
		\item проверка на среднее арифметическое значение последовтельности производных;
		\item проверка на непереодичность последовательности.
	\end{enumerate}
	\begin{equation}
		C = k_{avg} * c_{avg} + k_{avg_d }* c_{avg_d} + k_{period} * c_{period}
	\end{equation}
	\quad Где $C$ --- общий критерий, $k_{avg}$, $k_{avg_d }$, $k_{period}$ --- коэффициенты перечисленных критериев, $c_{avg}$,  $c_{avg_d}$, $c_{period}$ --- перечисленные критерии.
	

	
	\section*{Вывод}
	 \quad Использование комбинированного критерия позволяет обнаруживать не только отклонение от среднего значения последовательности, но и монотонность и периодичность последовательности.
	
	\clearpage
	
\end{document}
