\documentclass[12pt,a4paper,oneside]{report}
\usepackage[utf8]{inputenc}
\usepackage[english,russian]{babel}
\usepackage{amsmath}
\usepackage{amssymb}
\usepackage{geometry}
\usepackage{sverb}
\usepackage{graphicx}
\usepackage{pdfpages}
\usepackage{hyperref} 
\usepackage{url}
\usepackage{titlesec, blindtext, color}
\usepackage{listings}
\usepackage{pgfplots}
\pgfplotsset{compat=newest}
\graphicspath{{../}}
\DeclareGraphicsExtensions{.pdf,.png,.jpg}
\usepackage{tabularx}
\usepackage{subcaption}
\usepackage{colortbl}
\usepackage{multirow}
\usepackage{longtable}
\usepackage{enumitem}
\usepackage{algorithm}
\usepackage{tikz}
\usepackage[noend]{algpseudocode}
\usepackage{float}
\definecolor{gray75}{gray}{0.75}
\definecolor{Blue}{HTML}{5D8AA8}
\newcommand{\hsp}{\hspace{20pt}}

\newcommand{\RomanNumeralCaps}[1]
{\MakeUppercase{\romannumeral #1}}



% Для листинга кода:
\lstset{ %
	language=c,                 % выбор языка для подсветки (здесь это С)
	basicstyle=\small\sffamily, % размер и начертание шрифта для подсветки кода             % где поставить нумерацию строк (слева\справа)
	numberstyle=\tiny,           % размер шрифта для номеров строк
	stepnumber=1, 
	keywordstyle=\color{blue},% размер шага между двумя номерами строк
	numbersep=5pt,                % как далеко отстоят номера строк от подсвечиваемого кода
	showspaces=false,            % показывать или нет пробелы специальными отступами
	showstringspaces=false,      % показывать или нет пробелы в строках
	showtabs=false,             % показывать или нет табуляцию в строках
	frame=single,              % рисовать рамку вокруг кода
	tabsize=2,                 % размер табуляции по умолчанию равен 2 пробелам
	captionpos=t,              % позиция заголовка вверху [t] или внизу [b] 
	breaklines=true,           % автоматически переносить строки (да\нет)
	breakatwhitespace=false, % переносить строки только если есть пробел
	escapeinside={\#*}{*)}  % Стиль литералов
}


\titleformat{\chapter}[hang]{\Huge\bfseries}{\thechapter.\textcolor{gray75}\hsp}{0pt}{\Huge\bfseries}
\newcommand{\specchapter}[1]{\chapter*{#1}\addcontentsline{toc}{chapter}{#1}}
\newcommand{\specsection}[1]{\section*{#1}\addcontentsline{toc}{section}{#1}}
\newcommand{\specsubsection}[1]{\subsection*{#1}\addcontentsline{toc}{subsection}{#1}}

% геометрия


\setcounter{tocdepth}{4} % фикс переноса 
\righthyphenmin = 2
\tolerance = 2048


\thispagestyle{empty}

\geometry{pdftex, left = 3cm, right = 1cm, top = 2cm, bottom = 2cm}

\begin{document}
	
\thispagestyle{empty}


\noindent \begin{minipage}{0.15\textwidth}
	\includegraphics[width=\linewidth]{images/b_logo}
\end{minipage}
\noindent\begin{minipage}{0.85\textwidth}\centering
	\fontsize{12pt}{12pt}\selectfont
	\textbf{Министерство науки и высшего образования Российской Федерации}\\
	\textbf{Федеральное государственное автономное образовательное учреждение высшего образования}\\
	\textbf{«Московский государственный технический университет имени Н.Э.~Баумана}\\
	\textbf{(национальный исследовательский университет)»}\\
	\textbf{(МГТУ им. Н.Э.~Баумана)}
\end{minipage}

\noindent\rule{\linewidth}{3pt}
\newline\newline
\noindent ФАКУЛЬТЕТ $\underline{\text{«Информатика и системы управления»}}$ \newline\newline
\noindent КАФЕДРА $\underline{\text{«Программное обеспечение ЭВМ и информационные технологии»}}$

\vspace{1cm}

\begin{center}
	\noindent\begin{minipage}{1.3\textwidth}\centering
		\Large\textbf{Лабораторная работа № 4}\newline
		\textbf{по дисциплине <<Моделирование>>}\newline\newline
	\end{minipage}
\end{center}

\noindent\textbf{Тема} $\underline{\text{Моделирование системы массового обслуживания}}$\newline\newline
\noindent\textbf{Студент} $\underline{\text{Равашдех Ф.Х.}}$\newline\newline
\noindent\textbf{Группа} $\underline{\text{ИУ7-75Б}}$\newline\newline
\noindent\textbf{Преподаватель} $\underline{\text{Рудаков И.В..}}$\newline

\begin{center}
	\vfill
	Москва,~\the\year
\end{center}
\clearpage


\section*{Задание}
\quad Разработать программное обеспечение, моделирующее систему, состоящую из генератора, очереди сообщений и обслуживающего аппарата. Генератор подает в очередь сообщений заданное количество сообщений с интервалом согласно с заданным законом распределения. Обслуживающий аппарат работает с интервалом согласно с другим заданным законом распределения до тех пор, пока не будут обработаны все заявки, обработанные заявки могут снова попадать в очередь с заданной вероятностью. Определить длину очереди, при которой не будет потерянных сообщений.

\section*{Теория}

\subsection*{1. Пошаговый подход}

\quad Подход заключается в последоватеьном анализе состояний элементов системы в каждый момент времени $t + \Delta t$ по заданным состояниям элементов системы в момент $t$. При этом новое состояние элементов определяется в соответствии с их алгоритмическим описанием с учетом случайных факторов, заданных с использованием закона распределения. В результате такого анализа принимается решение о том, какие события системы необходимо имитировать в данный момент времени.


\subsection*{2. Событийный подход}

\quad Подход заключаетс в анализе состояний элементов системы в каждый момент изменения системы, т.е. при генерации и обработке сообщений в системе. Такой подход выгодно отличается от пошагового в том, что проводит анализ состояний элементов системы тогда и только тогда, когда это необхождимо, тем самым устранякет недостатки пошагового метода, заключающиеся в лишних анализах состояний, когда в системе ничего не меняется или в недостаточной обработке при большом шаге. Момент следующего анализа наступает не с приращением определенного интервала, а с наступлением следующего события, которое определяется минимальным значением времени из списка последующих событий.


\clearpage
	
\end{document}
