\addcontentsline{toc}{chapter}{СПИСОК ИСПОЛЬЗОВАННЫХ ИСТОЧНИКОВ}
\begin{thebibliography}{}
	\bibitem{lit2} Jannach D., Zanker M., Felfernig A., Friedrich G. Recommender Systems: An Introduction // Cambridge University Press, 2010. --- 352 с.
	\bibitem{lit1} Ricci F., Rokach L., Shapira B. Recommender Systems Handbook. 3-е изд. // New York: Springer, 2022. --- 1060 с.
	\bibitem{lit5} Zheng Y. Using Outlier Detection to Identify Grey-Sheep Users in Recommender Systems: A Comparative Study // Computers, Materials \& Continua. --- 2025. --- Vol. 83, № 3. --- P. 4315-4328.
	\bibitem{lit3} Qian Zhang, Jie Lu, Yaochu Jin. Artificial intelligence in recommender systems // Complex and Intelligent Systems. --- 2021. --- Vol. 7. --- P. 439-457.
	\bibitem{lit4} Zangerle E., Bauer C. Evaluating Recommender Systems: Survey and Framework // ACM Computing Surveys. --- 2022. --- Vol. 55, № 8. --- P. 1--38.
	\bibitem{lit11} Черняков А. Н., Дибиров М. Ш. О некоторых способах построения рекомендательных систем онлайн-маркетинга на основе алгоритмов машинного обучения // Инновации и инвестиции --- 2023 --- № 6.
	\bibitem{lit12} Волин М. В., Вовк Л. П. Рекомендательная система с контентной фильтрацией в электронной коммерции // Вестник науки и образования. --- 2025. №6 (161)-1.
	\bibitem{lit13} Михайлов А. Н. Разработка рекомендательных систем: методы и алгоритмы для электронной коммерции // Вестник науки. --- 2024. --- № 12 (81).
	\bibitem{lit14} Штовбонько А. Искусственный интеллект в электронной коммерции: алгоритмы рекомендаций и их влияние на поведение пользователей // Вестник науки. --- 2025. --- № 4 (85).
\end{thebibliography}
