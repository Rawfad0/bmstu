\chapter{Анализ предметной области}

\section{Основные определения}

Рекомендательные системы~---~это программные инструменты и методы, предоставляющие пользователю предложения относительно объектов, которые могут быть ему полезны~\cite{lit1}. В случае маркетплейсов такими предлагаемыми объектами будут являться товары. 

Обратная связь пользователя~---~реакция пользователя собираемая системой для использования в качестве обучения для рекомендательных систем~\cite{lit1}.

Обратную связь пользователя разделяют на два вида:
\begin{enumerate}
	\item Явная обратная связь~---~пользователь явно оценил объекты;
	\item Неявная обратная связь~---~обратная связь извлекается на основе наблюдения и анализа действий пользователя.
\end{enumerate}

Проблема нового пользователя или товара (проблема <<холодного старта>>)~---~ситуация, при которой невозможно точно вычислить рекомендации для новых пользователей или товаров, так как недостаточно информации о пользователе или товаре~\cite{lit4}.

Пользователи <<серые овцы>> (grey-sheep users)~---~пользователи с уникальными или нестандартными предпочтениями, которые ведут к низким корреляциям с большинством пользователей и, как следствие, приводят к нетипичным рекомендациям для них~\cite{lit5}.

\section{Формализация задачи}

Пусть $U$~---~множество пользователей, $I$~---~множество объектов. Для каждого пользователя $u \in U$ существует множество объектов $I_u \subset I$, с которыми он взаимодействовал, и которым он присвоил рейтинги согласно выражению~\ref{eq:1}.
\begin{equation}
	\label{eq:1}
	R = \left( r_{ui} \right)_{i \in I_u},
\end{equation}
где $R$~---~матрица отношения размером $n_{users}$~$\times$~$n_{items}$, $r_{ui}$~---~характеристика обратной связи пользователя $u$ с объектом $i$.  

Задача рекомендательных систем формулируется следующим образом: для каждого пользователя $u \in U$ необходимо оценить значения $r_{ui}$ для объектов $i \in I \setminus I_u$, выбрать заданное количество $k$ объектов с наибольшими предсказанными рейтингами $\hat{r}_{ui}$, что записывается в виде выражения~\ref{eq:2}.
\begin{equation}
	\label{eq:2}
	Recommend_k(u,\dots)=(i_1,\dots,i_k)
\end{equation}
где $Recommend_k$~---~функция, принимающая пользователя $u$, для которого выполняется рекомендация, и необходимые для конкретного метода параметры, и возвращающая упорядоченный набор $(i_1,\dots,i_k)$ из $k$ объектов, с наибольшими предсказанными рейтингами $\hat{r}_{ui}$ для пользователя $u$.

\clearpage
