\chapter{Методы решения}

\section{Основные методы и их классификация}

Методы рекомендаций обычно классифицируют следующим образом~\cite{lit1, lit2, lit11, lit13, lit14}:
\begin{enumerate}
	\item коллаборативная фильтрация;
	\item на основе содержания;
	\item на основе знания;
	\item гибридные.
\end{enumerate}

\subsection{Методы коллаборативной фильтрации}

Основная идея коллаборативных рекомендательных подходов заключается в использовании информации о прошлом поведении или мнениях существующего сообщества пользователей для прогнозирования того, какие объекты, вероятнее всего, будут наиболее релевантны текущему пользователю~\cite{lit2}.

Чистые подходы коллаборативной фильтрации не используют и не требуют информации о самих товарах и пользователях. Преимуществом этого метода является то, что эти данные не нужно вводить в систему или поддерживать.

Но данный метод обладает и недостатками, выделяют следующий ряд проблем характерных для коллаборативной фильтрации: 
\begin{enumerate}
	\item проблема нового пользователя;
	\item проблема нового товара;
	\item проблема информационного пузыря;
	\item проблема <<серой овцы>>;
	\item невозможность явно задать предпочтения;
\end{enumerate}

\subsection{Методы на основе содержания}

Рекомендации на основе содержимого строятся на наличии описаний объектов и профиля, который присваивает этим характеристикам определённое значение. Для маркетплейса возможными характеристиками товара могут быть категория, бренд, описание~\cite{lit2}. 
Подобно описаниям объектов, профили пользователей также могут автоматически формироваться и обучаться либо на основе анализа поведения и обратной связи пользователей, либо путём прямого опроса об их предпочтениях, например, через заполнение анкеты при регистрации на маркетплейсе.

По сравнению с подходами, не использующими информацию о содержании, рекомендации на его основе имеют два преимущества. Во-первых, им не требуются большие группы пользователей для достижения достаточной точности рекомендаций. Во-вторых, новые объекты можно сразу рекомендовать, как только становятся доступны их характеристики~\cite{lit2}. Таким образом, у данных методов отсутствуют проблемы нового пользователя, нового товара, <<серой овцы>>, при этом поддерживается высокий уровень персонализации. Основными недостатками этого метода являются низкая точность при малом количестве данных о характеристиках товаров или пользователях и проблема возникновения информационного пузыря, так как пользователь начнёт взаимодействовать с рекомендованными товарами, что повлечёт ещё большую рекомендацию подобных товаров~\cite{lit12}.

\subsection{Методы на основе знаний}

В рекомендательных системах, основанных на знаниях, рекомендации формируются на основе существующих знаний или правил о потребностях пользователей и функциях объектов. В отличие от методов, основанных на содержании или коллаборативной фильтрации, системы, основанные на знаниях, хранят базу знаний, созданную на основе информации, извлечённой из предыдущих записей пользователя. Эта база знаний содержит сведения о прошлых задачах, ограничениях и соответствующих решениях. Знания из базы используются системой при возникновении новой задачи рекомендации. Метод рассуждений на основе прецедентов применяет предыдущие случаи для решения текущей задачи и является широко используемой техникой в системах на основе знаний. В отличие от систем рекомендаций, основанных на содержании, поиск сходства между продуктами требует более структурированного представления. При этом осуществляется сравнение предыдущего случая с текущим и адаптация решения~\cite{lit3}.

Применение методов рекомендаций на основе знаний особенно ценно в областях, которые характеризуются высокой специфичностью знаний предметной области, и каждый случай представляет собой уникальную ситуацию. Одним из преимуществ этого подхода является отсутствие проблемы нового товара и проблемы нового пользователя, так как предварительные знания уже приобретены и сохранены в базе знаний. Ещё одним преимуществом является возможность пользователей накладывать ограничения на результаты рекомендаций. Однако у данного метода есть и недостатки --- дополнительные затраты на настройку и управление системой при создании и поддержке базы знаний~\cite{lit3}.

\subsection{Гибридные методы}

Гибридные рекомендательные системы представляют собой технические подходы, которые комбинируют несколько реализаций алгоритмов или компонентов рекомендаций~\cite{lit2}. 

На результат гибридной рекомендательной системы влияет архитектура системы и используемые методы рекомендательных систем. Таким образом, они приобретают необходимые свойства описанных ранее методов, что позволяет получить рекомендательные системы, которые минимально подвержены недостаткам, свойственные  другим методам, при этом обладая их преимуществами. 


\clearpage
