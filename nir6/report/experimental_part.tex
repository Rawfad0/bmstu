\chapter{Сравнение методов}

Для сравнения методов были сформулированы следующие критерии~\cite{lit2, lit3}:

\begin{enumerate}
	\item отсутствие проблемы нового пользователя;
	\item отсутствие проблемы нового товара;
	\item учёт параметров товара;
	\item учёт параметров пользователя;
	\item не требует параметров товара;
	\item не требует параметров пользователя;
	\item отсутствие проблемы информационного пузыря;
	\item высокая персонализация;
	\item отсутствие проблемы <<серой овцы>>;
	\item возможность явно задать предпочтения.
\end{enumerate}


В таблице~\ref{t:table} представлено сравнение перечисленных методов по сформулированным критериям.

\renewcommand{\arraystretch}{1.5}
\begin{table}[!ht]
	\begin{center}
		\small
		\begin{threeparttable}
			\caption{Сравнение классов методов}
			\label{t:table}
			\begin{tabular}{|l|c|c|c|c|}
				\hline
				\diagbox{\makecell{\textbf{Номер}\\\textbf{критерия}}}{\textbf{Метод}} & \makecell{\textbf{Коллаборативная}\\\textbf{фильтрация}} & \makecell{\textbf{На основе}\\\textbf{содержания}} & 
				\makecell{\textbf{На основе}\\\textbf{знаний}} & 
				\textbf{Гибридный} \\
				\hline
				1 & - & + & + & + \\
				\hline
				2 & - & + & + & + \\
				\hline
				3 & - & + & + & $\pm$ \\
				\hline
				4 & - & $\pm$ & - & $\pm$ \\
				\hline
				5 & + & - & - & $\pm$ \\
				\hline
				6 & + & + & + & $\pm$ \\
				\hline
				7 & - & - & + & $\pm$ \\
				\hline
				8 & + & $\pm$ & - & + \\
				\hline
				9 & - & + & - & $\pm$ \\
				\hline
				10 & - & $\pm$ & + & $\pm$ \\
				\hline
			\end{tabular}
		\end{threeparttable}
	\end{center}
\end{table}

%\subsection*{Вывод}

Гибридный метод удовлетворяет наибольшему количеству сформулированных для сравнения критериев, так как объединяет в себе другие методы для достижения необходимых свойств. Так, например, недостатки метода коллаборативной фильтрации компенсируются методом на основе содержания.

\clearpage
