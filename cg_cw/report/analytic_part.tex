\chapter{Аналитический раздел}

\section{Описание объектов сцены}

Сцена состоит из следующих объектов:
\begin{enumerate}
	\item камера --- объект сцены, из которого происходит наблюдение сцены, определяется местоположением и направлением просмотра;
	\item источник света --- объект сцены, который представляет из себя материальную точку, излучающую свет;
	\item модель --- трехмерный объект, расположенный на сцене, представленный в виде полигональной сетки;
	\item облако точек --- набор векторов вида $(x, y, z)$.
\end{enumerate}

\section{LiDAR}

LiDAR (Light Detection and Ranging) — это технология, которая использует лазерное излучение для обнаружения и определения дальности до объектов. Она применяется в разных областях, например, в метеорологии, геодезии, картографии и в системах машинного зрения.

Технология основана на принципе измерения времени прохождения лазерного импульса (ToF — Time of Flight). Лазерный излучатель генерирует короткие световые импульсы в инфракрасном или видимом диапазоне. Отражённые импульсы фиксируются датчиком, который измеряет время их возврата. Вычислительный модуль обрабатывает собранные данные и преобразует их в цифровую информацию.

Лидар работает с высокой частотой, испуская тысячи импульсов в секунду. Результаты объединяются в облака точек, представляющие собой трёхмерную модель.

В системах машинного зрения LiDAR используется для создания двумерных или трёхмерных изображений окружающего пространства.

\subsection{Шаблоны сканирования}

Чтобы собрать как можно больше данных, в LiDAR устанавливаются системы вращающихся зеркал, чтобы собрать как можно больше данных, не изменяя положения.
Различные алгоритмы, по которым вращаеются зеркала, образуют различные шаблоны (паттерны) сканирования --- то, как движется луч при сканировании.
Распространены следующие шаблоны сканирования:
\begin{itemize}
	\item вертикальные полосы --- сканирование происходит вертикально, от основания объекта до его вершины;
	\item горизонтальные полосы --- сканирование происходит горизонтально, слева направо или справа налево;
	\item сетка --- совмещение двух предыдущих шаблонов образует сканирование в виде сетки;
	\item с интервалом --- сканирование происходит с определёнными интервалами.
\end{itemize}

\section{Способы описания моделей}
Модель является отображением формы и размеров объектов. Основное назначение модели - правильно отображать форму и модели объекта.

В основном используются 3 вида модели~\cite{lit1, lit13}:
\begin{enumerate}
	\item каркасная (проволочная);
	\item поверхностная;
	\item объемная.
\end{enumerate}

\subsection{Каркасная модель}

В этой модели задается информация о вершинах и рёбрах объектов. Этим видам модели присущ один, но весьма существенный недостаток: не всегда модель правильно передает представление об объекте.

\subsection{Поверхностная модель}

Поверхность может описываться аналитически, либо может задаваться другим способом. Сложные криволинейные поверхности можно представлять в упрощенном виде, например, с помощью полигональной аппроксимации. В этом случае такая поверхность будет задаваться в виде поверхности многогранника.

Недостаток: отсутствует информация о том, с какой стороны поверхности находится собственный материал, а с какой стороны - пустота. 

\subsection{Объемная модель}

Отличие от поверхностной модели состоит только в том, что в объемных модели к информации о поверхности добавляется информация о том, где расположен материал. Проще всего это можно сделать путём указания направления внутренней нормали.

\subsection{Выбор модели}

Т.к. LiDAR сканирует поверхности объектов, то каркасная модель не подходит из-за отсутствия поверхностей. Объёмная модель отличается от поверхностной наличием информации о нахождении материала, что не влияет на сканирование, поэтому не имеет значения для данной задачи. Поэтому в результате выбора модели была выбрана поверхностная модель.


\section{Способы задания объектов}
%Для описания формы поверхности могут использоваться следующие способы.
%Поверхностная модель задается следующими способами:
Способы задания поверхностных моделей~\cite{lit1, lit12}:

\begin{itemize}
	\item аналитический --- описание поверхности с помощью математических уравнений;
	\item полигональный --- описания с использованием следующих элементов: вершины, отрезки прямых (векторы), полилинии, полигоны, полигональные поверхности;
\end{itemize}

Аналитичиский способ задания объекта ограничен невозможностью описывать произвольные объекты, что делает его неподходящим для решения задачи. Поэтому в результате был выбран полигональный способ задания объектов.

\section{Анализ алгоритмов удаления невидимых линий и поверхностей}

Задача удаления невидимых линий и поверхностей является одной
из наиболее сложных в машинной графике. Алгоритмы удаления
невидимых линий и поверхностей служат для определения линий
ребер, поверхностей или объемов, которые видимы или невидимы
для наблюдателя, находящегося в заданной точке пространства~\cite{lit1}.

\subsection{Алгоритм Робертса}

Алгоритм Робертса --- алгоритм, разработанный для удаления рёбер и граней, которые скрываются другими объектами. Алгоритм работает в пространстве объекта и сравнивает взаимное расположение поверхностей каждого тела.

\subsection{Алгоритм художника}

Алгоритм художника --- алгоритм решения проблемы <<видимости>> в компьютерной графике. Он позволяет избежать дополнительных затрат памяти, изначально сортируя по расстоянию от точки обзора. Затем объекты проверяются в порядке глубины, начиная с самого дальнего. В этом случае при рассмотрении объекта нет необходимости проверять его z-координату, так как цвет записывается в буфер кадра. Значения, хранящиеся в буфере ранее, перезаписываются.

\subsection{Алгоритм Z-буффера}

Алгоритм z-буфера --- это метод удаления невидимых частей объектов в трёхмерной сцене. В нём используется два буфера: буфер кадра и z-буфер. В буфере кадра хранится информация о цвете объекта для каждого пикселя, а в z-буфере --- z-координата видимого объекта.
Алгоритм с импользованием Z-буффера по шагам:
\begin{enumerate}
	\item заполнить буфер кадра фоновым цветом.
	\item  заполнить z-буфер минимальным значением z (определяет самую удалённую границу сцены).
	\item обработать каждый многоугольник в сцене: преобразовать его в растровую форму и вычислить глубину z для каждого пикселя.
	\item сравнить глубину каждого пикселя с z-буфером: если z больше, чем в z-буфере, записать атрибуты цвета многоугольника в буфер кадра и обновить z-буфер. Если сравнение даёт противоположный результат, ничего не делать.
\end{enumerate}

\subsection{Алгоритм трассировки лучей}

Алгоритм трассировки лучей предполагает отслеживание лучей света в обратном направлении, от наблюдателя к объекту. Каждый луч проходит через пиксель растра до сцены. Траектория каждого луча отслеживается, чтобы определить, какие именно объекты сцены, если таковые существуют, пересекаются с данным лучом. Необходимо проверить пересечение каждого объекта сцены с каждым лучом. Пересечение с ближайшим объектом представляет видимую поверхность для данного пикселя.

%\subsection{Сравнение алгоритмов}

%Для сравнения алгоритмов удаления невидимых линий и поверхностей были сформулированы следующие критерии:
%\begin{enumerate}
%	\item поддерживает 
%	\item
%	\item временная сложность
%	\item работает в пространстве изображения
%$\end{enumerate}

\subsection{Выбор алгоритма}

LiDAR работает на основе отражения лазерных импульсов от объектов, поэтому для решения задачи симуляции LiDAR был выбран алгоритм трассировки лучей, т.к. он моделирует распространение лучей в пространстве и их взаимодействие с поверхностями объектов.

%В результате сравнения алгоритмов были выбраны два алгоритма: Z-буффера и трассировки лучей. 


% Для каждого пикселя изображения формируется луч, который оп Каждый луч проходит через пиксель растра до сцены и отслеживает траекторию, определяя объекты, которые пересекаются с ним. Пересечения упорядочиваются по глубине, и видимая поверхность определяется по максимальному значению z.


%\section{}

%\section*{Вывод}

%<что сделали в аналите, кратко>

\clearpage
