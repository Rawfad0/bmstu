\chapter{Конструкторский раздел}


\section{Требования к программному обеспечению}

Разрабатываемое программное обеспечение должно предоставлять пользователю следующие возможности:
\begin{itemize}
	\item добавление объекта на сцену (куб, четырехугольная пирамида, тетраэдр, шестиугольная призма);
	\item удаление объектов со сцены;
	\item изменение свойств выбранного объекта (положение в пространстве, поворот);
	\item выбор камеры;
	\item задание шаблона и параметра сканирования;
	\item задание сценария сканирования;
	\item добавление и удаление облаков точек;
	\item перемещение и поворот камеры с помощью клавиатуры.
\end{itemize}

При этом разрабатываемая программа должна удовлетворять следующему требованию:
\begin{itemize}
	\item наличие визуализации облака точек.
\end{itemize}

\newpage

\section{Алгоритм трассировки лучей}
 
На рисунке~\ref{fig:rt_scheme} представлена схема алгоритма трассировки лучей. 

\begin{figure}[h]
	\centering
	\includesvg[scale=0.73]{/Users/rf9/PycharmProjects/cg_cw/report/images/rt.drawio.svg}
	\caption{Схема алгоритма трассировки лучей }
	\label{fig:rt_scheme}
\end{figure}

%\section{Алгоритм изображения облака точек}

%На рисунке~\ref{fig:rt_scheme} представлена схема алгоритма изображения облака точек. 

%\begin{figure}[h]
%	\centering
%	\includesvg[scale=0.9]{/Users/rf9/PycharmProjects/cg_cw/report/images/pc.drawio.svg}
%	\caption{Схема алгоритма изображения облака точек}
%	\label{fig:pc_scheme}
%\end{figure}

\newpage

\section{Типы данных}

В таблице~\ref{tbl:data_presentation} представлены структуры данных и их поля.
\begin{table}[h]
	\begin{center}
		\caption{Выбор структур данных для представления объектов}
		\label{tbl:data_presentation}
		\begin{tabular}{|l|l|}
			\hline
			\textbf{Объект}                & \textbf{Представление}                                                                                                      \\ \hline
			Сцена & Объект Scene:\\&
				 --- objs --- массив объектов сцены;\\&
				 --- point\_clouds --- массив облаков точек.\\
			\hline
			
			 Объект & Объект MeshModel:\\&
				 	--- points --- массив точек;\\&
				 	--- triangles --- массив треугольников;\\&
				 	--- name ---  наименование объекта (в таблице объектов).\\
			 \hline
			 
			 Облако точек & Объект PointCloud:\\&
			 --- points --- массив точек.\\
			 \hline
			 
			Камера & Объект Camera: \\&
				 	--- center --- центр камеры;\\&
				 	--- h\_dir --- горизонтальный вектор задания направления камеры;\\&
				 	--- v\_dir --- вертикальный вектор задания направления камеры;\\&
				 	--- h\_fov --- угол обзора по горизонтали;\\&
				 	--- v\_fov --- угол обзора по векртикали;\\&
				 	--- min\_z --- минимальная дальность отрисовки;\\&
				 	--- max\_z --- максимальная дальность отрисовки.\\

			\hline
			
			Полотно & Объект Canvas\\&
				--- width --- ширина полотна;\\&
				--- height --- высота полотна.\\
			\hline

		\end{tabular}
	\end{center}
\end{table}


%\section*{Вывод}

%<что сделали в конструкторке и получили в результате, кратко>

\clearpage
