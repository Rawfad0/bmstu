\ssr{ВВЕДЕНИЕ}

В современном мире технологии играют важную роль в различных сферах деятельности человека. Одной из таких технологий является LiDAR (Light Detection and Ranging), которая активно применяется в разных областях, например, в метеорологии, геодезии, картографии, автомобильной промышленности, робототехнике, в системах машинного зрения и других областях. 

\textbf{Цель курсовой работы} --- разработка программного обеспечения для симуляции LiDAR и построение 3D-карты сцены.
Для достижения поставленной цели необходимо выполнить следующие задачи:
\begin{enumerate}
	\item описать объекты и выбрать модель их представления;
	\item изучить и провести анализ существующих алгоритмов компьютерной графики; \item выбрать наиболее подходящие алгоритмы для реализации программы;
	\item спроектировать программное обеспечение, предоставляющее пользователю необходимые функции;
	\item реализовать спроектированное программное обеспечение;
	\item провести сравнительный анализ времени построения кадра в зависимости от удаления объекта сцены от камеры.
	
\end{enumerate}

\clearpage
