\chapter{Исследовательский раздел}

%<цель исследования, на какой машине делали (указать ЦПУ, ОЗУ, ОС), желательно написать, как исследовали и при каких условиях; что получили в результате (таблицы + графики)>

\section{Технические характеристики}

Технические характеристики устроства, на котором выполнялось тестирование:
\begin{itemize}
	\item операционная система --- macOS Sonoma 14.1 (23B2073);
	\item процессор --- Apple M3~\cite{lit3};
	\item оперативная память --- 16 Гб.
\end{itemize}

Замеры проводилось на ноутбуке, включенном в сеть питания, нагруженным процессами операционной системы, приложениями окружения и замеряемой программой.

\section{Описание исследования}

В ходе исследования необходимо проанализировать зависимость времени генерации кадра размером $800 \times 600$ от расстояния от камеры до объекта. На сцене содержится только один, заданный исследованием объект. Расстояние от камеры до объекта означает расстояние между точкой центра камеры и средней точкой всех точек объекта (центром масс объекта).

Исследование проводилось на следующих объектах: 
\begin{itemize}
	\item куб (12 треугольников);
	\item квадратная правильная пирамида (8 треугольников);
	\item треугольная правильная пирамида (тетраэдр) (4 треугольника);
	\item шестиугольная правильная призма (20 треугольников).
\end{itemize}

Результаты измерений для различных объектов представлены в таблицах~\ref{tbl:t_1}-\ref{tbl:t_4}.

\begin{table}[h]
	\begin{center}
		\caption{\label{tbl:t_1} Результаты измерений времени построения одного кадра в зависимости от расстояния до куба}
		\begin{tabular}{|r|r|}
			\hline
			\textbf{\begin{tabular}[c]{@{}l@{}}Расстояние до объекта\end{tabular}} & \textbf{\begin{tabular}[c]{@{}l@{}}Время построения одного \\ кадра в миллисекундах\end{tabular}} \\ \hline
			4 & 792.751 \\ \hline
			5 & 593.561 \\ \hline
			6 & 385.714 \\ \hline
			7 & 269.575 \\ \hline
			8 & 201.955 \\ \hline
			9 & 160.791 \\ \hline
			10 & 154.974 \\ \hline
			11 & 139.017 \\ \hline
			12 & 124.865 \\ \hline
			13 & 113.250 \\ \hline
			14 & 92.610 \\ \hline
			15 & 83.913 \\ \hline
			16 & 93.297 \\ \hline
			17 & 81.428 \\ \hline
			18 & 82.049 \\ \hline
			19 & 71.318 \\ \hline
			20 & 68.965 \\ \hline
		\end{tabular}
	\end{center}
\end{table}

\begin{table}[h]
	\begin{center}
		\caption{\label{tbl:t_2} Результаты измерений времени построения одного кадра в зависимости от расстояния до квадратной пирамиды}
		\begin{tabular}{|r|r|}
			\hline
			\textbf{\begin{tabular}[c]{@{}l@{}}Расстояние до объекта\end{tabular}} & \textbf{\begin{tabular}[c]{@{}l@{}}Время построения одного \\ кадра в миллисекундах\end{tabular}} \\ \hline
			4 & 327.318 \\ \hline
			5 & 240.370 \\ \hline
			6 & 161.174 \\ \hline
			7 & 140.256 \\ \hline
			8 & 118.707 \\ \hline
			9 & 102.096 \\ \hline
			10 & 93.984 \\ \hline
			11 & 85.103 \\ \hline
			12 & 82.478 \\ \hline
			13 & 73.347 \\ \hline
			14 & 71.728 \\ \hline
			15 & 69.722 \\ \hline
			16 & 63.068 \\ \hline
			17 & 51.587 \\ \hline
			18 & 62.824 \\ \hline
			19 & 64.616 \\ \hline
			20 & 62.851 \\ \hline
		\end{tabular}
	\end{center}
\end{table}

\begin{table}[h]
	\begin{center}
		\caption{\label{tbl:t_3} Результаты измерений времени построения одного кадра в зависимости от расстояния до тетраэдра}
		\begin{tabular}{|r|r|}
			\hline
			\textbf{\begin{tabular}[c]{@{}l@{}}Расстояние до объекта\end{tabular}} & \textbf{\begin{tabular}[c]{@{}l@{}}Время построения одного \\ кадра в миллисекундах\end{tabular}} \\ \hline
			4 & 193.817 \\ \hline
			5 & 145.702 \\ \hline
			6 & 133.799 \\ \hline
			7 & 112.043 \\ \hline
			8 & 98.981 \\ \hline
			9 & 89.180 \\ \hline
			10 & 81.717 \\ \hline
			11 & 76.380 \\ \hline
			12 & 71.771 \\ \hline
			13 & 69.786 \\ \hline
			14 & 67.180 \\ \hline
			15 & 63.685 \\ \hline
			16 & 64.003 \\ \hline
			17 & 63.105 \\ \hline
			18 & 61.160 \\ \hline
			19 & 56.520 \\ \hline
			20 & 57.133 \\ \hline
		\end{tabular}
	\end{center}
\end{table}

\begin{table}[h]
	\begin{center}
		\caption{\label{tbl:t_4} Результаты измерений времени построения одного кадра в зависимости от расстояния до шестиугольной призмы}
		\begin{tabular}{|r|r|}
			\hline
			\textbf{\begin{tabular}[c]{@{}l@{}}Расстояние до объекта\end{tabular}} & \textbf{\begin{tabular}[c]{@{}l@{}}Время построения одного \\ кадра в миллисекундах\end{tabular}} \\ \hline
			4 & 1353.018 \\ \hline
			5 & 1362.413 \\ \hline
			6 & 1341.379 \\ \hline
			7 & 1091.668 \\ \hline
			8 & 852.148 \\ \hline
			9 & 641.851 \\ \hline
			10 & 491.618 \\ \hline
			11 & 391.221 \\ \hline
			12 & 322.493 \\ \hline
			13 & 274.486 \\ \hline
			14 & 237.936 \\ \hline
			15 & 211.138 \\ \hline
			16 & 185.139 \\ \hline
			17 & 171.488 \\ \hline
			18 & 141.959 \\ \hline
			19 & 146.449 \\ \hline
			20 & 132.270 \\ \hline
		\end{tabular}
	\end{center}
\end{table}

% Замер времени количества пересечения луча с треугольником
На рисунке~\ref{fig:plot1} представлен графики зависимости времени построения кадра от расстояния камеры до объекта.

\begin{figure}[h]
	\centering
	\includesvg[scale=0.9]{/Users/rf9/PycharmProjects/cg_cw/report/images/Figure_2.svg}
	\caption{Графики зависимости времени построения кадра от расстояния камеры до объекта}
	\label{fig:plot1}
\end{figure}
%\begin{figure}[h]
%	\centering
%	\includesvg[scale=0.9]{/Users/rf9/PycharmProjects/cg_cw/report/images/Figure_1_md.svg}
%	\caption{Графики зависимости времени построения кадра от расстояния камеры до объекта}
%	\label{fig:plot1}
%\end{figure}

\section*{Вывод}

Исследование показало:
\begin{itemize}
	\item время построения кадра уменьшается при удалении объекта от камеры;
	\item время построения кадра увеличивается при увеличении числа треугольников, из которых состоит объект.
\end{itemize}


%<что сделали и получили в результате, кратко>

\clearpage
