\chapter{Методы решения}

\section{Основные методы и их классификация}

Методы решения задачи многокритериальной оптимизации обычно классифицируются на следующие классы методов:
\begin{enumerate}
	\item методы, не учитывающие предпочтения (no-preference);
	\item априорные (a priori) методы;
	\item апостериорные (a posteriori) методы;
	\item интерактивные (interactive) методы.
\end{enumerate}

\subsection{Методы, не использующие предпочтения}

Методам данного класса не требуется никакой информации о предпочтениях ЛПР, как только определены ограничения и цели задачи. Таким образом, такой подход требует, чтобы ЛПР могло принять решение, полученное с помощью метода. Преимущество этого пути заключается в том, что в процессе получения решения аналитик не будет беспокоить ЛПР, что предпочтительнее с точки зрения ЛПР. Но основным недостатком в этом случае является необходимость для аналитика делать множество предположений о предпочтениях ЛПР. Это трудно сделать даже самому лучшему и знающему аналитику~\cite{lit11}.

Примером данного класса является метод глобального критерия~\cite{lit2, lit11}. 
В этом методе расстояние между некоторой контрольной точкой и достижимой целевой областью минимизируется. Необходимо выбрать точку отсчета и метрику для измерения расстояний. Все целевые функции считаются одинаково важными~\cite{lit12}.

\subsection{Априорные методы}

Априорные методы многокритериальной оптимизации представляют собой класс методов, которые используют предварительные знания о предпочтениях ЛПР для определения оптимальных решений. Эти методы основываются на формализации принципов оптимальности и позволяют интегрировать различные критерии в единое решение. Чаще всего предварительную информацию формализуют таким образом, чтобы свести многокритериальную задачу к однокритериальной~\cite{lit2}.

Наиболее распространенным подходом к построению решающего правила на основе предпочтений ЛПР является построение функции полезности, полностью отражающей предпочтения ЛПР по отношению к величинам частных критериев. В таком случае поиск решения сводится к нахождению допустимого решения, которое максимизирует значение функции полезности~\cite{lit1}.

Примеры методов данного класса:
\begin{enumerate}
	\item метод полезной (утилитарной) функции;
	\item лексикографический метод;
	\item скаляризация (скалярная свертка);
	\item метод целевого программирования.
\end{enumerate}

\subsection{Апостериорные методы}

Апостериорные методы также называются методами генерации оптимальных по Парето решений. После того как набор оптимальных по Парето решений (или его часть) сформирован, он представляется ЛПР, которое выбирает наиболее предпочтительный из альтернатив~\cite{lit2, lit12}. 

Недостаток метода заключается в том, что равномерная аппроксимация множества и/или фронта Парето обычно требует больших вычислительных затрат.
Другим недостатком, который серьезно ограничивает практическую применимость этих методов, является то, что они обычно генерируют большое количество недоминируемых решений, поэтому для ЛПР становится практически невозможным выбрать наиболее удовлетворительное из них~\cite{lit11, lit12}.

Наконец, возникает самостоятельная проблема визуализации фронта Парето для задач с числом критериев большим двух~\cite{lit2}.

Примеры методов данного класса:
\begin{enumerate}
	%\item метод анализа иерархий;
	\item параметрический метод (метод взвешивания (weighting));
	\item метод $\varepsilon$-ограничений.
\end{enumerate}

\subsection{Интерактивные методы}

В интерактивных методах многокритериальной оптимизации процесс принятия решения является интерактивным, и ЛПР активно взаимодействует с методом на протяжении всего поиска наиболее предпочтительного решения~\cite{lit13}.

\newpage

В интерактивных методах обычно используются следующие шаги~\cite{lit13}:
\begin{enumerate}
	\item инициализация (например, вычисление идеального и приближенного к минимуму (nadir) целевых векторов и представление их ЛПР);
	\item генерация оптимальной отправной точки по Парето (например, используя какой-либо метод без предпочтений, предложенные ЛПР);
	\item запрос у ЛПР информации о предпочтениях;
	\item генерация новых оптимальных решений по Парето в соответствии с предпочтениями ЛПР и представление их ЛПР;
	\item если было сгенерировано несколько решений, выбор лицом, принимающем решение, наилучшего на данный момент решения;
	\item если ЛПР принимает решение остановиться, то остановка, иначе --- переход в пункт (3). 
\end{enumerate}



\clearpage
