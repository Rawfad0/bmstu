\ssr{ВВЕДЕНИЕ}


% Маркетплейс --- это онлайн-платформа, объединяющая на одной цифровой площадке множество продавцов и покупателей. Основная задача маркетплейса заключается в том, чтобы предоставить продавцам удобные инструменты для размещения и продвижения своих товаров, а покупателям --- единый интерфейс для поиска, выбора и приобретения продукции. В отличие от традиционных интернет-магазинов, маркетплейс не ограничивается товарами одного поставщика, что обеспечивает широкое разнообразие ассортимента и конкурентные цены.

Маркетплейс --- это онлайн-платформа, основная задача которой заключается в предоставлении продавцам удобных инструментов для размещения и продвижения своих товаров, а покупателям --- единый интерфейс для поиска, выбора и приобретения продукции. В отличие от традиционных интернет-магазинов, маркетплейс не ограничивается товарами одного поставщика, что обеспечивает широкое разнообразие ассортимента и конкурентные цены.

Актуальность маркетплейсов обусловлена стремительным развитием электронной коммерции и цифровых сервисов. Современные пользователи всё чаще совершают покупки в интернете, и спрос на универсальные торговые платформы постоянно растёт. Маркетплейсы становятся важным элементом экономической экосистемы, так как упрощают взаимодействие между продавцами и покупателями, способствуют развитию малого и среднего бизнеса, а также формируют новые бизнес-модели.

Функционирование маркетплейса невозможно без хранения и обработки больших объёмов информации. В базе данных маркетплейса хранятся сведения о товарах, пердложениях, пользователях, заказах и отзывах. 

%Эффективная организация и оптимизация структуры базы данных обеспечивает быстродействие системы, целостность данных и удобство работы конечных пользователей, что делает данную область особенно значимой при проектировании информационных систем маркетплейсов.

%Хранение и обработка запросов в централизованной базе данных позволяет турагентствам эффективно управлять увеличивающимся объемом бронирований и различных туров.


\textbf{Цель курсовой работы} --- разработка программного обеспечения маркетплейса, которое позволит продавцам размещать предложения о продаже товаров, а покупателям находить интересующие товары и покупать их.

% где продавцы смогут предлагать свои товары, а покупатели — находить интересующие их товары с подходящими условиями. 

% хранить подробную информацию для выбора по параметрам и выявлять наиболее выгодные предложения для клиентов.

Для достижения поставленной цели необходимо выполнить следующие задачи:

\begin{enumerate}
	\item провести анализ предметной области и существующих баз данных;
	\item спроектировать базу данных и программное обеспечение;
	\item реализовать спроектированные базу данных и программное обеспечение;
	\item провести исследование разработанной базы данных.
\end{enumerate}

\clearpage
