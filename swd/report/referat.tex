\begin{center}
	\LARGE\bfseries{РЕФЕРАТ}
\end{center}

Расчетно-пояснительная записка \pageref{LastPage} с., \totalfigures{} рис., \totaltables{} таблиц, 13 источников, 1 приложение.

БАЗА ДАННЫХ, СИСТЕМА УПРАВЛЕНИЯ БАЗАМИ ДАННЫХ, РЕЛЯЦИОННАЯ БАЗА ДАННЫХ, МАРКЕТПЛЕЙС.

Цель работы — разработка программного обеспечения маркетплейса, которое позволит продавцам размещать предложения о продаже товаров, а покупателям находить интересующие товары и покупать их.

Был проведен анализ предметоной области, сравнительный анализ существующих решений и анализ существующих баз данных. Была формализована задача и требования к разрабатываемой базе данных и приложению. Была проведена формализация данных и ролей в базе данных и приложении. Было проведено описание сущностей базы данных, описание атрибутов, типов данных и ограничений, проектирование базы данных, описание ролевой модели. В данном разделе был проведен выбор средств реализации приложения, языка программирования, системы управления базой данных. Были приведены архитектура приложения, листинги кода для создания таблиц базы данных и листинги взаимодействия приложения с базой данных. Исследование продемонстрировало, что оптимизация запросов за счёт использования индексов — с учётом структуры запросов и особенностей данных — позволяет сократить время их выполнения.