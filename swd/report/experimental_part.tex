\chapter{Исследовательский раздел}

%<цель исследования, на какой машине делали (указать ЦПУ, ОЗУ, ОС), желательно написать, как исследовали и при каких условиях; что получили в результате (таблицы + графики)>


%<что сделали и получили в результате, кратко>

\section{Технические характеристики}

Характеристики устройства, на котором выполнялись замеры~\cite{lit21}:

\begin{enumerate}
	\item операционная система --- macOS Sonoma 14.1 (23B2073);
	\item процессор --- Apple M3;
	\item оперативная память --- 16 Гб.
\end{enumerate}

\clearpage

\section{Цель исследования}

Целью исследования является сравнение времени выполнения запроса \textit{GetOfferDetailsByProductId} с индексом~\textit{idx\_offers\_product\_id} и без него в зависимости от количества строк в таблице~\textit{offers}.

\clearpage

\section{Результаты исследования}

В таблице~\ref{tab:index_performance} представлены результаты сравнения врмени выполнения запроса с индексом и без него в зависимости от количества строк в таблице~\textit{offers}.

\begin{table}[H]
	\centering
	\caption{Сравнение времени выполнения запроса с индексом и без него}
	\label{tab:index_performance}
	\begin{tabular}{|c|c|c|}
		\hline
		\textbf{Количество строк} & \textbf{Без индекса (мс)} & \textbf{С индексом (мс)} \\ \hline
		10\,000  & 5.0764  & 1.4626 \\ \hline
		30\,000  & 6.8016   & 0.9772 \\ \hline
		50\,000  & 8.3237   & 0.7982 \\ \hline
		75\,000  & 11.8892  & 0.7729 \\ \hline
		100\,000 & 12.0410  & 0.8758 \\ \hline
		150\,000 & 14.7833  & 0.6418 \\ \hline
		200\,000 & 18.2809  & 0.4954 \\ \hline
		250\,000 & 16.9305  & 0.6610 \\ \hline
		300\,000 & 19.9839  & 0.6416 \\ \hline
		400\,000 & 27.2926  & 0.6148 \\ \hline
	\end{tabular}
\end{table}

На рисунке~\ref{fig:figure1} представлены графики зависимости времени врмени выполнения запроса с индексом и без него в зависимости от количества строк в таблице~\textit{offers}.

\begin{figure}[h!]
	\centering
	\includesvg[scale=0.6]{images/Figure_2.svg}
	\caption{Графики зависимости времени выполнения запроса с индексом и без него от количества строк}
	\label{fig:figure1}
\end{figure}

\clearpage

\section*{Вывод}

Из проведенного исследования следует, что наличие индекса (по умолчанию \textit{B-tree}) подходящего под частый запрос (запрос предложений --- один из основных запросов маркетплейса) уменьшает время выполнения запроса относительно запроса без индекса (обычный последовательный поиск \textit{sequential scan}), тем самым ускоряя работу приложения и повышая отзывчивость для пользователя, что улучшает его пользовательский опыт.

\clearpage
