\chapter{Исследовательский раздел}

%<цель исследования, на какой машине делали (указать ЦПУ, ОЗУ, ОС), желательно написать, как исследовали и при каких условиях; что получили в результате (таблицы + графики)>

В данном разделе будет приведены результаты проведенного нагрузочного тестирования разработанного сервера по максимальному количеству обслуживаемых сетевых соединений и скорости отдачи данных по каждому сетевому соединению и совокупной.

\section{Технические характеристики}

Характеристики устройства, на котором выполнялись замеры~\cite{lit21}:

\begin{enumerate}
	\item операционная система --- macOS Sonoma 14.1 (23B2073);
	\item процессор --- Apple M3;
	\item оперативная память --- 16 Гб.
\end{enumerate}

\section{Цель исследования}

Целью исследования является провести нагрузочное тестирование разработанного сервера:
\begin{enumerate}
	\item Максимальное количество обслуживаемых сетевых соединений
	\item Скорость отдачи данных по каждому сетевому соединению и совокупная
\end{enumerate}

\clearpage

\section{Результаты исследования}

\subsection{Максимальное количество обслуживаемых сетевых соединений}


В таблице~\ref{tab:load_test_simple} представлены результаты нагрузочного тестирования максимальное количество обслуживаемых сетевых соединений для разного количества процессов prefork.

\renewcommand{\arraystretch}{1.5}
\begin{table}[!ht]
	\begin{center}
		\small
		\begin{threeparttable}
			\caption{Нагрузочное тестирование сервера при различном числе процессов}
			\label{tab:load_test_simple}
			\begin{tabular}{|c|c|c|c|c|}
				\hline
				\textbf{Процессов} & \textbf{Соединений} & \textbf{Успех} & \textbf{Неуспех} & \textbf{\%} \\
				\hline
				2 & 16 & 16 & 0 & 100 \\
				2 & 32 & 32 & 0 & 100 \\
				2 & 64 & 64 & 0 & 100 \\
				2 & 128 & 128 & 0 & 100 \\
				2 & 256 & 128 & 128 & 50 \\
				2 & 512 & 130 & 382 & 25.3906 \\
				2 & 1024 & 245 & 779 & 23.9258 \\
				\hline
				4 & 16 & 16 & 0 & 100 \\
				4 & 32 & 32 & 0 & 100 \\
				4 & 64 & 64 & 0 & 100 \\
				4 & 128 & 128 & 0 & 100 \\
				4 & 256 & 129 & 127 & 50.3906 \\
				4 & 512 & 128 & 384 & 25 \\
				4 & 1024 & 264 & 760 & 25.7812 \\
				\hline
				8 & 16 & 16 & 0 & 100 \\
				8 & 32 & 32 & 0 & 100 \\
				8 & 64 & 64 & 0 & 100 \\
				8 & 128 & 128 & 0 & 100 \\
				8 & 256 & 128 & 128 & 50 \\
				8 & 512 & 166 & 346 & 32.4219 \\
				8 & 1024 & 383 & 641 & 37.4023 \\
				\hline
			\end{tabular}
		\end{threeparttable}
	\end{center}
\end{table}

На рисунке~\ref{fig:fig1} представлены графики зависимости процента успешных соединений от количества соединений для различного числа процессов.

\begin{figure}[h!]
	\centering
	\includesvg[scale=0.5]{images/1.svg}
	\caption{Графики зависимости процента успешных соединений от количества соединений для различного числа процессов}
	\label{fig:fig1}
\end{figure}


\subsection{Скорость отдачи данных по каждому сетевому соединению и совокупная}



В таблице~\ref{tab:throughput_test} представлены результаты нагрузочного тестирования скорости отдачи данных для различного числа процессов prefork. 

\renewcommand{\arraystretch}{1.5}
\begin{table}[!ht]
	\begin{center}
		\small
		\begin{threeparttable}
			\caption{Совокупная скорость отдачи данных сервера при различном числе процессов}
			\label{tab:throughput_test}
			\begin{tabular}{|c|c|c|c|c|}
				\hline
				\textbf{Процессов} & \textbf{Соединений} & \textbf{МБ} & \textbf{Время (c))} & \textbf{Скорость (МБ/с)} \\
				\hline
				2 & 4 & 4.00024 & 0.00650454 & 614.992 \\
				2 & 8 & 8.00048 & 0.00949479 & 842.618 \\
				2 & 16 & 16.001 & 0.015737 & 1016.77 \\
				2 & 32 & 32.0019 & 0.0232845 & 1374.39 \\
				2 & 64 & 64.0038 & 0.0367464 & 1741.77 \\
				2 & 100 & 100.006 & 0.0591705 & 1690.13 \\
				\hline
				4 & 4 & 3.00018 & 0.00339233 & 884.4 \\
				4 & 8 & 8.00048 & 0.00708058 & 1129.92 \\
				4 & 16 & 16.001 & 0.0116171 & 1377.36 \\
				4 & 32 & 32.0019 & 0.0244327 & 1309.8 \\
				4 & 64 & 64.0038 & 0.0232089 & 2757.73 \\
				4 & 100 & 100.006 & 0.0418988 & 2386.85 \\
				\hline
				8 & 4 & 4.00024 & 0.00475004 & 842.149 \\
				8 & 8 & 8.00048 & 0.006594 & 1213.3 \\
				8 & 16 & 16.001 & 0.0123947 & 1290.95 \\
				8 & 32 & 32.0019 & 0.0240008 & 1333.37 \\
				8 & 64 & 64.0038 & 0.0224389 & 2852.36 \\
				8 & 100 & 100.006 & 0.0381876 & 2618.81 \\
				\hline
			\end{tabular}
		\end{threeparttable}
	\end{center}
\end{table}

На рисунке~\ref{fig:fig3} представлены графики зависимости скорости отдачи данных от количества соединений для различного числа процессов на соединение.  

\begin{figure}[h!]
	\centering
	\includesvg[scale=0.5]{images/3.svg}
	\caption{Графики скорости отдачи данных сервера от количества соединений для различного числа процессов на соединение}
	\label{fig:fig3}
\end{figure}

На рисунке~\ref{fig:fig2} представлены графики зависимости совокупной скорости отдачи данных от количества соединений для различного числа процессов.  

\begin{figure}[h!]
	\centering
	\includesvg[scale=0.5]{images/2.svg}
	\caption{Графики совокупной скорости отдачи данных сервера от количества соединений для различного числа процессов}
	\label{fig:fig2}
\end{figure}


\clearpage

\section*{Вывод}

В результате проведённого нагрузочного тестирования была исследована совокупная скорость отдачи данных сервером при различном числе процессов prefork и различном количестве одновременных сетевых соединений.

Установлено, что при малом числе соединений увеличение количества процессов приводит к росту совокупной пропускной способности сервера. Так, при 4-16 соединениях сервер с 4 и 8 процессами демонстрирует более высокую скорость передачи данных по сравнению с конфигурацией из 2 процессов. Это связано с уменьшением времени ожидания обработки запросов и более эффективным использованием вычислительных ресурсов.

При увеличении числа соединений до 64 наблюдается максимальная совокупная скорость отдачи данных. Наиболее высокие значения достигаются при использовании 8 процессов prefork, где скорость передачи превышает 2800~MБ/s. Это указывает на то, что при достаточном уровне параллелизма сервер способен эффективно масштабироваться по числу обслуживаемых соединений.

При дальнейшем увеличении числа соединений до 100 рост совокупной скорости замедляется. Данный эффект объясняется ограничениями системных ресурсов, таких как пропускная способность сети, контекстные переключения между процессами и накладные расходы на обслуживание большого количества одновременных соединений.

Таким образом, увеличение числа процессов prefork положительно влияет на совокупную скорость отдачи данных до определённого предела. 

\clearpage
