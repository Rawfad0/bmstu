\chapter{Конструкторский раздел}

В данном разделе будут рассмотрены алгоритмы работы сервера и работы prefork.
%В данном разделе будут спроектированы базы данных, описаны сущности, роли и триггеры. 

\section{Алгоритм работы HTTP-сервера}

Алгоритм работы сервера для отдачи статического содержимого по протоколу HTTP включает следующие основные этапы:

\begin{enumerate}
	\item Инициализация серверного сокета: создание сокета, установка параметров, привязка к заданному порту и перевод сокета в режим прослушивания.
	\item Инициализация структуры для хранения активных клиентских соединений.
	\item Ожидание сетевых событий с использованием механизма мультиплексирования ввода-вывода.
	\item При появлении события на слушающем сокете сервер принимает новое клиентское соединение и переводит соответствующий сокет в неблокирующий режим.
	\item При готовности клиентского сокета к чтению сервер считывает входные данные и анализирует метод и путь к запрашиваемому ресурсу.
	\item В случае неподдерживаемого метода формируется ответ с кодом состояния 405.
	\item При запросе поддерживаемого метода определяется путь к файлу в каталоге статического содержимого и выполняется проверка существования и прав доступа к ресурсу.
	\item Если файл не найден, формируется ответ с кодом состояния 404; при отсутствии прав доступа — ответ с кодом состояния 403.
	\item При успешной проверке формируется HTTP-ответ с кодом состояния 200 и заголовками ответа.
	\item При использовании метода GET сервер передаёт содержимое файла клиенту частями, учитывая ограничения размера буфера.
	\item При использовании метода HEAD сервер передаёт только заголовки ответа без тела сообщения.
	\item После завершения передачи данных сервер закрывает клиентское соединение и освобождает ресурсы.
	\item Сервер возвращается к ожиданию новых сетевых событий.
\end{enumerate}

\section{Алгоритм работы модели prefork}

Алгоритм работы сервера с использованием модели prefork состоит из следующих этапов:

\begin{enumerate}
	\item Основной процесс сервера выполняет инициализацию сетевого окружения и создаёт слушающий сокет.
	\item После успешной инициализации основной процесс создаёт фиксированное количество дочерних процессов с помощью системного вызова \texttt{fork}.
	\item Каждый дочерний процесс наследует слушающий сокет и переходит в основной цикл обработки клиентских соединений.
	\item Дочерние процессы независимо друг от друга ожидают сетевые события и принимают входящие соединения.
	\item После принятия соединения дочерний процесс обрабатывает запрос клиента в соответствии с алгоритмом работы HTTP-сервера.
	\item Каждый дочерний процесс может одновременно обслуживать несколько клиентских соединений с использованием механизма мультиплексирования ввода-вывода.
	\item Основной процесс после создания дочерних процессов переходит в режим ожидания и не участвует в обработке клиентских запросов.
	\item В случае завершения работы одного из дочерних процессов основной процесс может создать новый процесс для поддержания заданного количества рабочих процессов.
\end{enumerate}

\section{Архитектура сервера}

Разрабатываемый HTTP-сервер построен на основе архитектуры prefork, при которой главный процесс сервера отвечает за инициализацию и создание набора дочерних процессов, предназначенных для обработки клиентских соединений. Каждый дочерний процесс функционирует независимо и готов принимать входящие соединения от клиентов.

Для обслуживания нескольких сетевых соединений в рамках одного процесса используется механизм мультиплексирования ввода-вывода select(), позволяющий отслеживать состояние множества файловых дескрипторов и реагировать на события ввода без блокировки выполнения процесса. Такой подход позволяет одному процессу эффективно обрабатывать несколько клиентских запросов одновременно.

Выбранная архитектура обеспечивает изоляцию процессов, повышает надёжность сервера и упрощает обработку ошибок, так как аварийное завершение одного дочернего процесса не приводит к остановке всего приложения. Кроме того, использование стандартных POSIX-механизмов обеспечивает переносимость решения между UNIX-подобными операционными системами.


\section*{Вывод}

В данном разделе были рассмотрены алгоритмы рассмотрены алгоритмы работы сервера и работы prefork, а также архитектура сервера.

% описание сущностей базы данных, описание атрибутов, типов данных и ограничений, проектирование базы данных, описание ролевой модели.

%<что сделали в конструкторке и получили в результате, кратко>

\clearpage
