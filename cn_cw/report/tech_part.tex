\chapter{Технологический раздел}
 
\section{Выбор инструментов}

Разработанный HTTP-сервер реализован на языке C++ с использованием стандартных POSIX-механизмов для сетевого взаимодействия и мультиплексорной обработки соединений. В качестве компилятора использовался clang 15.0.0 с поддержкой стандартов языка C++17~\cite{lit11}. Разработка велась в редакторе Visual Studio Code с настройками для C/C++.


\section{Структура проекта}

Проект разделён на модули.
\begin{enumerate}
	\item main.cpp~---~очка входа, создание слушающего сокета и запуск процессов;
	\item http\_server.cpp/h — обработка соединений, мультиплексирование через select()~\cite{lit12}, разбор HTTP-запросов, формирование ответов, поддержка GET/HEAD и кодов 200, 403, 404, 405~\cite{lit13};
	\item log.cpp/h — логирование запросов с указанием метода, пути и кода ответа~\cite{lit14}.
\end{enumerate}

\section{Тестирование}

В рамках проверки корректности работы разработанного HTTP-сервера было проведено функциональное тестирование, включающее проверку обработки HTTP-методов, кодов ответа сервера, а также корректности передачи данных.

\subsection{Проверка обработки HTTP-запросов}

Для проверки базовой функциональности сервера использовалась утилита \texttt{curl}. При выполнении HTTP GET-запроса к корневому ресурсу сервера (\texttt{/}) сервер корректно вернул код ответа \texttt{200 OK} и передал содержимое главной HTML-страницы. Полученный результат подтверждает корректную обработку запросов на получение существующего ресурса.

Для проверки поддержки метода HEAD был выполнен запрос заголовков без получения тела ответа. Сервер вернул код \texttt{200 OK} и корректный заголовок \texttt{Content-Length}, при этом тело ответа отсутствовало, что соответствует требованиям протокола HTTP.

При обращении к файлу, для которого отсутствуют права на чтение, сервер корректно определил невозможность доступа к ресурсу и вернул код ответа \texttt{403 Forbidden}. Это подтверждает корректную проверку прав доступа к файлам на стороне сервера.

Запрос несуществующего ресурса привёл к возврату кода ответа \texttt{404 Not Found}, что свидетельствует о корректной обработке ситуации отсутствия запрашиваемого файла в корневом каталоге сервера.

При попытке выполнения запроса с использованием неподдерживаемого HTTP-метода (POST) сервер вернул код ответа \texttt{405 Method Not Allowed}. Таким образом, сервер корректно ограничивает набор поддерживаемых методов и соответствует требованиям спецификации HTTP.

\subsection{Проверка корректности передачи данных}

Для проверки целостности передаваемых данных был выполнен тест загрузки файла размером 128 Мбайт. После загрузки файла контрольная сумма SHA-256 загруженного файла была сравнена с контрольной суммой исходного файла, расположенного на сервере. Полученные контрольные суммы полностью совпали, что подтверждает корректную передачу данных без потерь и искажений.


\section*{Вывод}

В технологическом разделе описаны используемые инструменты и среда разработки, структура проекта, тестирование, что обеспечивает систематичное и эффективное покрытие всех сценариев работы сервера.

\clearpage
