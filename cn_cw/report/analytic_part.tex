\chapter{Аналитический раздел}

В этом разделе будет перечислены теоретические сведения, необходимые для решения поставленной задачи.
%В этом разделе будет проведен анализ предметоной области, существующих решений, системных вызовов и способов реализации. Будет формализована задача и требования к разрабатываемому серверу.



\section{Протокол передачи гипертекста}

HTTP (HyperText Transfer Protocol, протокол передачи гипертекста)~---~протокол прикладного уровня, предназначенный для передачи данных между клиентом и сервером по модели <<запрос–ответ>>. Протокол является текстовым и не хранит состояние (stateless).
HTTP-сообщение состоит из стартовой строки, набора заголовков и, при необходимости, тела сообщения.

Методы, необходимые для отдачи статического содержимого:

\begin{enumerate}
	\item \textbf{GET} используется для запроса ресурса, расположенного на сервере, при успешной обработке запроса сервер возвращает заголовки и тело ответа, содержащее запрошенный файл;
	\item \textbf{HEAD} аналогичен методу \texttt{GET}, однако сервер возвращает только заголовки ответа без передачи тела сообщения, данный метод используется для получения информации о ресурсе без его загрузки.
\end{enumerate}

В результате обработки запросов сервер возвращает следующие коды состояния HTTP:

\begin{enumerate}
	\item \textbf{200 OK}~---~запрос выполнен успешно;
	\item \textbf{403 Forbidden}~---~доступ к ресурсу запрещён;
	\item \textbf{404 Not Found}~---~запрашиваемый ресурс не найден;
	\item \textbf{405 Method Not Allowed}~---~метод запроса не поддерживается сервером.
\end{enumerate}


\section{Веб-сервер}

Веб-сервер~---~это программное обеспечение, принимающий HTTP-запросы от клиентов и выдающий им HTTP-ответы.
Веб-серверы обладают возможностью предоставлять клиентам доступ к статическому содержимому из хранилища сервера, для этого клиенту требуется отправлить запрос к веб-серверу, который находит необходимый файл, считывает его и отсылает клиенту.
Веб-серверы также могут являтся связующим звеном между клиентами, например браузерами, и серверами-приложенями, проксируя трафик.

\section{Существующие решения}

Для отдачи статического содержимого в основном используются уже существующие решения.

Наиболее распространёнными веб-серверами являются:

\begin{enumerate}
	\item Apache HTTP Server;
	\item Nginx;
	\item Lighttpd.
\end{enumerate}

Apache HTTP Server~---~свободный веб-сервер, наиболее часто используемый в UNIX-подобных операционных системах. Основными достоинствами Apache считаются надёжность и гибкость конфигурации. Он позволяет подключать внешние модули для предоставления данных, использовать СУБД для аутентификации пользователей, модифицировать сообщения об ошибках и т. д.

Nginx~---~позиционируется производителем как простой, быстрый и надёжный сервер, не перегруженный функциями. Применение nginx целесообразно прежде всего для статических веб-сайтов и как обратного прокси-сервера перед динамическими сайтами.

Lighttpd~---~свободный веб-сервер, разрабатываемый с расчётом на скорость и защищённость, соответствие стандартам и с небольшим размером. Сложнее в конфигурации чем другие. Часто используется во встроенных системах, например в маршрутизаторах.

В рамках курсовой работы разрабатывается упрощённый HTTP-сервер, реализующий только необходимую функциональность, что позволит на практике изучить принципы работы веб-серверов.


\section{Сокеты}

Сокет~---~это абстракция конечной точки соединения, которая используется для обеспечения обмена данными между устройствами сети. Сокеты являются ключевым компонентом для установки и управления сетевыми соединениями.
Сокеты предоставляют интерфейс для создания конечных точек соединения в сети с использованием протоколов передачи данных. При разработке веб-серверов сокеты используются для «прослушивания» входящих соединений от клиентов и передачи данных между сервером и клиентами.

Процесс создания сервера с использованием сокетов включает в себя следующие шаги.

\begin{enumerate}
	\item создание сокета, который будет слушать входящие соединения;
	\item привязка сокета к адресу и порту, на котором будет прослушиваться входящий трафик;
	\item установка сервера в состояние прослушивания входящих соединений;
	\item принятие входящего соединения;
	\item обработка запросов и передача данных. 
\end{enumerate}

Использование сокетов в разработке серверов позволяет эффективно управ-
лять сетевыми соединениями и обеспечивать передачу статического контента клиентам по сети.

\section{Мультиплексирование}

Мультиплексирование ввода-вывода является методом обработки нескольких операций ввода-вывода в одном потоке выполнения программы, что позволяет повысить эффективность управления множеством соединений в сетевых приложениях или приложениях с асинхронным вводом-выводом. Это позволяет уменьшить задержки и повысить производительность.

Мультиплексоры позволяют приложению ожидать ввода или вывода данных из нескольких источников, таких как сокеты, файлы или сетевые устройства, используя один системный вызов вместо создания или управления каждым потоком или процессом отдельно.

Использование мультиплексирования ввода-вывода требует более сложной логики обработки событий, чем простое параллельное программирование, но оно может обеспечить более эффективное использование системных ресурсов и более высокую производительность.

Существуют следующие механизмы мультиплексирования сетевых соединений:

\begin{enumerate}
	\item select;
	\item pselect;
	\item poll;
	\item epoll (kqueue).
\end{enumerate}

\clearpage


Вызов \textbf{select}~---~это системный вызов, используемый в различных операционных системах для ожидания событий на нескольких файловых дескрипторах, таких как сокеты~\cite{lit1}. Он позволяет одному потоку контролировать несколько соединений одновременно. Данный мультиплексор имеет ограничение на максимальное количество отслеживаемых файловых дескрипторов (1024).

Вызов \textbf{pselect} схож по принципу работы с select и имеет те же ограничения, од- нако реализует более продвинутую обработку сигналов. В отличие от select, pselect позволяет процессу заблокироваться на ожидании событий ввода-вывода и одновременно игнорировать определенные сигналы или обрабатывать их в специфическим способом.

Вызов \textbf{poll}~---~это более новый метод опроса сокетов, созданный после того, как люди начали пытаться писать большие и высоконагруженные сетевые сервисы. Он спроектирован намного лучше и не страдает от большинства недостатков метода select, например от ограничения количества наблюдаемых дескрипторов.

Вызов \textbf{epoll (kqueue)}~---~истемный вызов, предоставляемый в ядре Linux, обеспечивает более гибкие варианты работы с событиями ввода-вывода за счёт реализации модели уведомлений. В отличие от select и pselect, epoll предоставляет более эффективный способ мониторинга множества файловых дескрипторов на предмет готовности к вводу или выводу и не имеет ограничений на обработку файловых дескрипторов.
Является наиболее современным и эффективным механизмом в сравнении с select и pselect и часто используется в высоконагруженных сетевых приложениях на ОС Linux~\cite{lit2}.


\section{Параллельная обработка запросов}

Для ускорения обработки запросов веб-серверы реализуют их параллельную обработку в многопроцессорной среде. 


В данной работе будут несколько основных существующих подходов к параллелизации работы веб-серверов:

\begin{enumerate}
	\item пул потоков (thread pool);
	\item разветвление (prefork)
\end{enumerate}


\textbf{Thread Pool} представляет собой набор потоков, готовых к выполнению задач. Когда в систему поступает новая задача, она помещается в очередь, и один из доступных потоков в пуле забирает эту задачу на выполнение. После завершения задачи поток возвращается обратно в пул и становится доступным для выполнения новых задач. Это способствует уменьшению накладных расходов на создание и уничтожение потоков, а также позволяет управлять ресурсами более эффективно и обеспечивает параллельное выполнение задач.

\textbf{Prefork}~---~это модель параллелизации, при которой веб-сервер создаёт отдельные процессы для обработки запросов. Такой подход позволяет изолировать обработку каждого запроса и повышает надежность веб-сервера, так как сбои в одном процессе не влияют на остальные. Однако создание процессов требует больше дополнительных ресурсов, чем использование пула потоков.

Выбор между подходами зависит от конкретных требований веб-сервера, предполагаемой нагрузки и характера обрабатываемых запросов.

\section*{Вывод}


В этом разделе были перечислены теоретические сведения, необходимые для решения поставленной задачи, такие как протокол передачи гипертекста, веб-сервер и его существующие решения, сокеты и механизмы мультиплексирования.

\clearpage
